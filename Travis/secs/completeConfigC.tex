

This algorithm is complete but unsound.
We first define the set of incorrect files based on ground truth $Inc_{gt}$, and the set of files which are predicted by the model to be incorrect, $Inc_{prd}$.
To define $Inc_{gt}$, we will introduce the set of relations that are necessary and breaking based on ground truth, $Nec_{gt}$ and $Br_{gt}$ respectivly.
The definition of $Inc_{prd}$ follows from the description of the algorithm.

\begin{align}
Inc_{gt} = \{&\forall C, \exists r \in Nec_{gt}, r \notin M(C)\\ \nonumber
 &\lor \exists r \in Br_{gt}, r \in M(C)\} \\
Inc_{prd} = \{&\forall C, \exists r \in Nec, r \notin M(C)\}
\end{align}

Our definition of sound and complete are as follows:

\begin{align}
&\forall C \in Inc_{gt}, C \in Inc_{prd} \text{  Complete} \\
&\not \exists C \in Inc_{prd}, C \notin Inc_{gt} \text{  Sound}
\end{align}

To show completness we show that the $Inc_{gt} \subseteq Inc_{prd}$.

\begin{align*}
Nec \subseteq Nec_{gt}
%Nec \disjoint Br_{gt}
%\forall C \in {Incorrect(C)}, C \in system(Inc,C)   Nec_{True}, r \in Nec_{Learned} - (Nec_{Learned} \contains Nec_{True})
%\not \exists C \in SysInc(C), C \notin {Incorrect(C)} 
%\Nec_{Learned} \contains Nec_{Observable}
%\forall r \in Br, r \notin Incorrect(C) \implies \exists r \in Nec_{True}, r \notin M(C)
%Incorrect(C) \implies \exists r \in Nec, r \notin M(C)
\end{align*}

%The algorithm is complete if, for any file which breaks a relation for which we have a template (a misconfiguration) and have seen an correct example, the system will mark that file as incorrect.
%Since the model always maintins the strongest conditions for correctness for any file, the algorithm is clearly complete.
%Completness cooresponds to no false negatives.
%This algorithm is however unsound, some correct file may be marked as incorrect, a false postive.
%We have a high false postive rate since the model is an overappoximation of the true model.
