\section{Version Space for Verification}

ConfigC~\cite{Santolucito2016} learns a classification model from a set of correct configuration files, then reports misconfigurations.
Specifically, ConfigC uses version space learing to identify errors in configuration files for MySQL.
Version space learning builds a logical constraint model for binary classification, or membership in a set~\cite{mitchell82}.
Traditional version space learning builds a model using a series of disjunctions from a set of predefined hypotheses.
By instead restricting the model to a series of conjunctions, ConfigC can not only flag misconfigurations, but also give the points of failure for non-membership.

Our definition of a configuration file is a file that can be transformed to an ordered list of $(keyword, value)$ pairs, called $Line$s.
To build the model, ConfigC take a single configuration file, $C$, and derives the set of all possible relations from each file, $M(C)$.
The classes of relations must be provided by the user as templates.
A template is a function taking some number of lines and determining a relation on the keywords of the lines ($Line^{n} \rightarrow Rel_{k1,..,kn}$).
In the case of ordering rules, this function will take two lines and indicate whether the first keyword should come before, after, or have no relation to the second. 
ConfigC features the following default rules.

\begin{center}
\begin{tabular} {|c |c |c |}
 \hline
 Template & Input  & Relation \\
\hline
 ordering & (Line,Line) & Before $\vert$ After $\vert$ None \\
 \hline
 integer relation & (Line,Line)  & $< | == | >$  \\
\hline
 type & (Line,Line)  & String $\vert$ Int $\vert$ Filepath $\vert$ IP  \\
\hline
 missing entry & (Line,Line) & Required $\vert$ Not \\
\hline
\end{tabular}
\end{center}

It is assumed if a file is correct, all relations in that file are in the set of necessary relations for any other file to be correct ($Correct(C) \implies \forall r \in M(C), r \in Nec$).
In this way, the initial model is built by creating the strongest conditions for a correct file, called the \textit{specific boundary} in version space learning.
This model is then iteratively relaxed as more examples are seen, a process called \textit{candidate elimination}.
Since we maintain the strongest condition for correctness, this approach will identify many correct files as incorrect, called a high false positive rate.
This is a problem when a user is then asked to manual review many files for errors, when the errors are not truly a problem.

When the system is run on a new file, one of the relations from the templates will not be satisfied.
To provide an error message,the system only needs to report which relation is not satisfied.

%To relax the model, two sets of relations from two files are merged into a single consistent set.
%\begin{lstlisting}
%n = {}
%for (c in files):
%  n1 = M(c)
%  n = merge(n, n1)
%\end{lstlisting}
