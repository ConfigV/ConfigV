\section{Limitations}

\subsection{Unsatisfiability}
Note we have made two assumptions in Eq. \ref{eq:E1}; first that the summary $P_t$ contains every piece of information that might lead to a build error, and second that the model $M(P_t)$ will learn all relations that might lead to a build error.
While these are the strong assumptions - our algorithm is able to detect cases where the summary is incomplete.
If we cannot find a solution for Eq. 3, it means either $P_t$ or $M(P_t)$ has been underspecified.
It is then the user's responsibility to expand the definitions accordingly.

\subsection{Trusted Base}
In addition to the completeness of $P_t$ and $M(P_t)$, we must pick a trusted base.
Since it is possible for TravisCI to have bugs, it is possible that between versions of TravisCI, two identical program summaries may have different build statuses.
Since version space learning is (generally) intolerant of noise, we require that
$P_t = P'_{t'} \implies S(P_t) = S(P'_{t'})$.
