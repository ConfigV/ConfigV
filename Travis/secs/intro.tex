\section{Introduction}

Machine learing has been used in various software analysis techniques, such as programming-by-example \cite{lau2000version}, invariant synthesis \cite{garg2014ice}, and error detection \cite{Santolucito2016}.
However, many popular machine learning algorithms, such as neural nets and n-gram models, are not designed to provide simple justifications for their classification results.
While effective in practice, the lack of justification for the results limits the applicability to software analysis, where justification is critical.
In the case of error detection, the system should not only report which files have errors, but also locate the errors.
Version space learning is a learning strategy for logical constraint classification~\cite{mitchell82}, which can be used to easily provide clear error messages.

%However, many popular machine learning algorithms, such as neural nets and n-gram models, produce probabilistic models of correctness.
%While effective in practice, these approachs cannot be garunteed to be complete, that is they will always find the error.
%We use version space learning to address the need for a formal completness garuntee in automated model generation for verification.

In our previous work we used version space learning to verify configuration files from a set of correct examples~\cite{Santolucito2016}.
The tool, ConfigC, takes examples of correct files, builds a language model, which is then used to check is new configuration files adhear to those rules.
If a file is incorrect, the tool will identify the location of error, what is incorrect.
For example, ConfigC might output \texttt{Ordering Error: Expected "extension mysql.so" before "extension recode.so"}.
We describe here this algorithm in terms of version space learning to show how clear error messages are produced with this technique.
%and show that the results are garunteed to be complete.

%Although the results are complete, the approach used previously produces a high false positive rate.
Although the error messages are clear, the approach used previously produces a high false positive rate (marks correct files as incorrect).
To decrease this, in this paper we propose an extention to the algorithm to handle sets of both incorrect and correct examples.
We also use a commonly avaible, but underutilized structure of learning sets for machine learning for software analysis.
Because most code does not exist in isolation, but changes over time with development, examples for a learning set can also leverage this temporal structure. 
This structure is simply a partial order, and any analysis that makes use of code from a version control system like Github will have this property.
We predict this algorithm has the potential to significantly reduce the false positive rate.
%A learning algorithm that uses the temporal structure of code to create sound classification with a low false positive rate can be used in at least the above listed software analysis techniques.

To test this approach in practice, we plan to implement our algorithm to check for TravisCI configuration errors.
TravisCI is a continuous integration tool connected to Github that allows programmers to automatically run their test suite on every code update (commit).
A user adds a configuration file to the repository that enables TravisCI and specifies build conditions, such as which compiler to use, which dependencies are required, and a set of benchmarks to test.
This ensures the tool can always be automatically built correctly on a fresh machine.

A recent usage study of TravisCI found that 15-20\% of failed TravisCI builds are due to "errors" - which means the configuration file was malformed and the software could not even be built \cite{API}.
Using the data from \cite{API}, we can also learn that since the start of 2014, approximately 88,000 hours of server time was used on TravisCI projects that resulted in an error status.
This number not only represents lost server time, but also lost developer time, as programmers must wait to verify that their work does not break the build.
If these malformed projects could be quickly statically checked on the client side, both TravisCI and its users could benefit.
