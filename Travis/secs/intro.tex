\section{Introduction}

Machine learing has been used in various software analysis techniques, such as programming-by-example \cite{lau2000version}, invariant synthesis \cite{garg2014ice}, and error detection \cite{Santolucito2016}.
However, many popular machine learning algorithms, such as neural nets and n-gram models, are not designed to provide simple justifications for their classification results.
While effective in practice, the lack of justification for the results limits the applicability to software analysis, where justification is often critical.
For example, in the case of error detection, the system should not only report which files have errors, but also locate the errors.
Version space learning is a technique for logical constraint classification~\cite{mitchell82}.
As a logical constraint system, version space learning can provide the justification that is difficult to obtain from other methods, but its main weakness is that it cannot handle noisy data.

%However, many popular machine learning algorithms, such as neural nets and n-gram models, produce probabilistic models of correctness.
%While effective in practice, these approachs cannot be garunteed to be complete, that is they will always find the error.
%We use version space learning to address the need for a formal completness garuntee in automated model generation for verification.

In our previous work we built a prototype to verify configuration files from a set of correct examples~\cite{Santolucito2016}.
The tool, ConfigC, takes as a training set a set of correct configuration files and builds a set of rules describing a language model.
This model is then used to check if new a configuration file adhears to those rules.
If a file is incorrect, the tool will identify the location of error, and report what is incorrect.
For example, ConfigC might output \texttt{Ordering Error: Expected "extension mysql.so" before "extension recode.so"}.
While this prototype's results are promising, the main weakness are that the learning can only use correct configuration files.
Additionally, there is a relatively high false positive rate (marking correct files as incorrect).

In order to extend the prototype, in this paper we provide a new description of ConfigC in terms of version space learning.
This description will allow us to understand how the justifications are produced in ConfigC.
We can then extend ConfigC with a new algorithm to decrease the false positive rate, while maintaning the ability to provide justification.
%and show that the results are garunteed to be complete.

%Although the results are complete, the approach used previously produces a high false positive rate.
We propose an extension to ConfigC that allows it to handle both correct and incorrect files in the training set.
Additionally, we extend ConfigC with the ability to use structue within the training set.
While the training set from ConfigC was a random sampling of configuration files, many other training sets have a structure based on version history.
%The algorithm we propose takes advantage of this commonly availible, but underutilized structure of training sets for software analysis.
%Because most code does not exist in isolation, but changes over time with development, any training set of code from a verison control system, like Github, has a rich temporal structure. 
This structure is simply a partial order, and can be used by our proposed algorithm for more effective learning.
%A learning algorithm that uses the temporal structure of code to create sound classification with a low false positive rate can be used in at least the above listed software analysis techniques.

To test this approach in practice, we plan to implement our algorithm to check for TravisCI configuration errors.
TravisCI is a continuous integration tool connected to Github that allows programmers to automatically run their test suite on every code update (commit).
A user adds a configuration file to the repository that enables TravisCI and specifies build conditions, such as which compiler to use, which dependencies are required, and a set of benchmarks to test.
This ensures the tool can always be automatically built correctly on a fresh machine.

A recent usage study~\cite{API} of TravisCI found that 15-20\% of failed TravisCI builds are due to "errors" - which is the TravisCI code used to mean the configuration file was malformed and the software could not even be built.
Using the data from \cite{API}, we can also learn that since the start of 2014, approximately 88,000 hours of server time was used on TravisCI projects that resulted in an error status.
This number not only represents lost server time, but also lost developer time, as programmers must wait to verify that their work does not break the build.
If these malformed projects could be quickly statically checked on the client side, both TravisCI and its users could benefit.

Our main contributions are then as follow:

\begin{enumerate}

\item We give a description of our prototype tool, ConfigC, in the context of version space learning.
\item We propose a new algorithm for ConfigC to handle both incorrect and correct training data, as well as internal structure of a training set.
\item We propose a real-world application of this approach, and outline the steps needed for an implementation.

\end{enumerate} 
