
%-----------------------------------------------------------------------------
%
%               Template for sigplanconf LaTeX Class
%
% Name:         sigplanconf-template.tex
%
% Purpose:      A template for sigplanconf.cls, which is a LaTeX 2e class
%               file for SIGPLAN conference proceedings.

%
% Guide:        Refer to "Author's Guide to the ACM SIGPLAN Class,"
%               sigplanconf-guide.pdf
%
% Author:       Paul C. Anagnostopoulos
%               Windfall Software
%               978 371-2316
%               paul@windfall.com
%
% Created:      15 February 2005
%
%-----------------------------------------------------------------------------

\documentclass{sigplanconf}

% The following \documentclass options may be useful:

% preprint      Remove this option only once the paper is in final form.

% 10pt          To set in 10-point type instead of 9-point.

% 11pt          To set in 11-point type instead of 9-point.

% authoryear    To obtain author/year citation style instead of numeric.

\usepackage{amsmath}
\usepackage{todonotes}
\usepackage{graphicx}\usepackage{color}
\usepackage{csquotes}


\begin{document}

\special{papersize=8.5in,11in}\setlength{\pdfpageheight}{\paperheight}\setlength{\pdfpagewidth}{\paperwidth}
\conferenceinfo{CONF 'yy}{Month d--d, 20yy, City, ST, Country} \copyrightyear{20yy} \copyrightdata{978-1-nnnn-nnnn-n/yy/mm} \doi{nnnnnnn.nnnnnnn}

% Uncomment one of the following two, if you are not going for the 
% traditional copyright transfer agreement.

%\exclusivelicense                
% ACM gets exclusive license to publish,                                   
% you retain copyright

%\permissiontopublish           <  
% ACM gets nonexclusive license to publish                                  
% (paid open-access papers,                                   
% short abstracts)
\titlebanner{banner above paper title}        
% These are ignored unless\preprintfooter{short description of paper}   
% 'preprint' option specified.

\title{Detecting TravisCI Misconfiguration}
\subtitle{Plan of Attack}

\authorinfo{Name1}           {Affiliation1}           {Email1}
\authorinfo{Name2\and Name3}           {Affiliation2/3}           {Email2/3}

\maketitle

\begin{abstract}
We can detect TravisCI misconfigurations using ConfigC.
\end{abstract}

%\category{CR-number}{subcategory}{third-level}

% general terms are not compulsory anymore, 
% you may leave them out\termsterm1, term2
%\keywords keyword1, keyword2

\section{Availability of Data}

TravisCI has recently released a metadata api to study their tool\cite{API}.

The first use of this API was analyzing the metadata of builds. It was found that \textasciitilde 15-20\% of failed TravisCI builds are due to "errors". My understanding of the API is that this corresponds to a misconfiguration. This is a large enough number that not only will reducing this be more convenient for users, but it could also significantly reduce server costs for Travis. By using the \verb|tr_duration| field from the API, we can also figure out how much server time we can save\cite{API}.


\section{Feasability Anaysis}
\label{sec:feas}

The language of TravisCI configuration files can likely be handled by ConfigC.
ConfigC works on languages with shallow parse trees, and most of the .travis.yml files is shallow.
However, because .travis.yml allows for bash scripts, some error may be out of scope.
We want to identify the frequency of the types of errors - if too many are related to poorly formed bash scripts, a naive application of ConfigC will not suffice.

To confirm that the types of most misconfigurations are in the scope of ConfigC, we should compile a collection of common misconfigurations. 
Usually such a task requires a domain expert.
However, in this case we may be able to learn the common errors by using the API and the temporal data from the commits.

We might code an error as the diff between a sequential pair of an erroring commit and a passing commit. 
As a new direction, we may also explore how useful this data is for learning/verification directly.
My guess is that is data will be too noisy, but either way, it should give us a sense of the problem.

To build such a database, scrape the .travis.yml files from each commit for a set of repos.
This can be achieved with the \verb|tr_status, git_commit and gh_project_name| fields from the API.

\section{Implementation}

To build the learning set, we will use the API to build tuples of .travis.yml and repo info. 
In addition to the .travis.yml file (collected as above), we will need at a minimum \verb|gh_lang and tr_status|. 
With this \verb|(File,gh_lang,tr_status,...)| tuple, we can then (almost) directly apply ConfigC to detect errors in the shallow part of the tree.

The first (and simplest) requirement is a .travis.yml parser, which is an extension of normal yml parsing to handle travis' ability to include bash scripts.
Second, depending on the results from Sec \ref{sec:feas}, we may also require some extra domain knowledge of the types of errors to be encoded as possible rules.
This is a slightly more involved task that requires writing some non-trivial Haskell code - but it really isn't that bad.


% We recommend abbrvnat bibliography style.

\bibliographystyle{abbrvnat}

% The bibliography should be embedded for final submission.
\bibliography{biblio}



\end{document}
