\documentclass{article}
\usepackage[utf8]{inputenc}

\title{Proposed Solution}
\author{}
\date{December 2015}

\usepackage{natbib}
\usepackage{graphicx}
\usepackage{listings}
\usepackage[usenames,dvipsnames]{color}

\definecolor{identifierColor}{rgb}{0.65,0.16,0.16}
\definecolor{comment_color}{rgb}{0.40,0.46,0.3}
\definecolor{num_color}{gray}{0.55}
\lstset{
  basicstyle=\footnotesize,
  breaklines=true,
  frame=bottomline,
  language=haskell,
  %identifierstyle=\color{identifierColor},
  morecomment=[l][\color{comment_color}\ttfamily]{--},
  backgroundcolor=\color{white},   % choose the background color; you must add \usepackage{color} or \usepackage{xcolor}
  breakatwhitespace=false,         % sets if automatic breaks should only happen at whitespace
  %captionpos=b,                    % sets the caption-position to bottom
  %commentstyle=\color{mygreen},    % comment style
  %frame=single,	                   % adds a frame around the code
  keepspaces=true,                 % keeps spaces in text, useful for keeping indentation of code (possibly needs columns=flexible)
  keywordstyle=\color{blue},       % keyword style
  otherkeywords={*,let, Server, Replication, FaultGraph, rankRCG, print, fialProb, goal, ...},           % if you want to add more keywords to the set
  numbers=left,                    % where to put the line-numbers; possible values are (none, left, right)
  numbersep=5pt,                   % how far the line-numbers are from the code
  numberstyle=\tiny\color{num_color}, % the style that is used for the line-numbers
  rulecolor=\color{black},         % if not set, the frame-color may be changed on line-breaks within not-black text (e.g. comments (green here))
  showtabs=false,                  % show tabs within strings adding particular underscores
  stepnumber=1,                    % the step between two line-numbers. If it's 1, each line will be numbered
  stringstyle=\color{mymauve},     % string literal style
  %title=\lstname                  % show the filename of files included with \lstinputlisting; also try caption instead of title
  mathescape=true,
  tabsize=3,
  literate=*{->}{{\textcolor{blue}{$\to$}}}{1}
           {<-}{{\textcolor{blue}{$\leftarrow$}}}{1}
}


\begin{document}

\maketitle

\section{Introduction}
In order to solve the misconfiguration problems outlined in \cite{yin2011}, we take a two stage process.

\begin{enumerate}
  \item Preprocessor
  \begin{enumerate}
    \item Given a set of config files in a language $[C_{httpd}]$, convert them to a set $[D]$ in our language.
    \item Learn constraints on syntax based on $[D]$ to generate the rule set $R_{httpd}$
  \end{enumerate}
  \item Usertime checking
  \begin{enumerate}
    \item convert a given configuration file $c_{httpd}$, into a file $d$, in our own language.
    \item check $d$ against $R_{httpd}$ and report errors to user.
  \end{enumerate}
\end{enumerate}

Expressed in pseduocode:

\begin{lstlisting}
main = 
  r = preproc learningSet
  e = usertime r userFile

preproc :: [ConfigFile a] -> (RuleSet,a)
preproc cs =
  ds = map convert cs 
  rules = learnOn ds
  fileType = getType cs
  return (rules,fileType)

convert :: ConfigFile a -> ConfigFile Common
getType :: [ConfigFile a] -> a
learnOn :: [ConfigFile Common] -> RuleSet

usertime :: (RuleSet,a) -> ConfigFile a -> Maybe Error
usertime r c =
  d = convert c
  e = verifyOn r d
  return e
\end{lstlisting}




\section{Preprocessor}
The first step is to use a \texttt{CONVERTER} to translate any config file type (httpd,MySQL,etc.) to a common format. 
Our \texttt{LEARNER} for learning constaints will depend on all files having a common format. 
The types of constraints, and the proposed solution to learn such constraints are outlined here.

\subsection{Format - Lexical}
This problem is exemplified in box (a) of the image.

\subsection{Format - Syntax}
This problem is exemplified in box (b) and (c) of the image.

\subsection{Value - Value Inconsistency}
This problem is exemplified in box (f) (g) and (h) of the image.

\section{Usertime Checking}


\bibliographystyle{plain}
\bibliography{references}
\end{document}

