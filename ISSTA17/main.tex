%% For double-blind review submission
\documentclass[sigconf]{acmart}

%% For single-blind review submission
%\documentclass[acmlarge,review]{acmart}\settopmatter{printfolios=true}
%% For final camera-ready submission
%\documentclass[acmlarge]{acmart}\settopmatter{}

%% Note: Authors migrating a paper from PACMPL format to traditional
%% SIGPLAN proceedings format should change 'acmlarge' to
%% 'sigplan,10pt'.


%% Some recommended packages.
\usepackage{booktabs}   %% For formal tables:
                        %% http://ctan.org/pkg/booktabs
\usepackage{subcaption} %% For complex figures with subfigures/subcaptions
                        %% http://ctan.org/pkg/subcaption

\usepackage{amsmath}

\newcommand{\cL}{{\cal L}}
\let\terms\undefined


\usepackage{times}            % standard fixed width font
\usepackage{graphicx}
\usepackage{amsmath}
\usepackage{xspace}
\usepackage{footnote}
\usepackage{cite}
\usepackage{amsfonts}
\usepackage{subfig}
%\usepackage{natbib}
\usepackage{hhline}
\usepackage{multirow}
\usepackage{setspace} 
\usepackage{epsfig}
\usepackage[hyphens]{url}
\usepackage[colorlinks,linkcolor=blue,citecolor=blue,urlcolor=blue]{hyperref}
\usepackage[hyphenbreaks]{breakurl}
\usepackage{booktabs}
%\usepackage[compact]{titlesec}
\usepackage{xcolor}
\usepackage[algoruled,vlined,ruled,linesnumbered]{algorithm2e}
\usepackage{lipsum}
\usepackage{courier}
\usepackage{listings}
\usepackage{mathpartir}
%\usepackage[scaled=0.78]{DejaVuSansMono}

\lstset{
  language=C,
	basicstyle=\footnotesize\ttfamily,
  breaklines=true,
  frame=single
}

%\usepackage[T1]{fontenc}
%\usepackage[scaled=0.78]{DejaVuSansMono}

\clubpenalty=10000      % penalty for creating a club line at end of line.
\widowpenalty=10000     % penalty for creating a widow line at top of page.

% Select one or other if want to see comments.
% \com is sometimes displayed during draft.
\long\def\com#1{}
%\long\def\com#1{{\bf \sc comment: }{\small [#1]}{\bf \sc\ endcomment}\newline}

\long\def\ennan#1{{\color{red}{\bf Ennan: }{\small [#1]}}}
\long\def\ruzica#1{{\color{red}{\bf Ruzica: }{\small [#1]}}}
%\long\def\xxx#1{}

% Use this macro to force page breaks where ugly widows/orphans occur;
% be sure to recheck all uses after any significant change to the text!
\def\widowpage{\pagebreak}

% Choose abbreviated or long-version alternatives in paper
%\long\def\abbr#1#2{#1}			% abbreviated version
\long\def\abbr#1#2{#2}			% long version

% Choose abbreviations or long names/titles in bibliography
%\def\bibbrev#1#2{#1}			% short version
%\def\bibbrev#1#2{#2}			% long version
\def\bibbrev#1#2{\abbr{#1}{#2}}		% follow abbr macro

% Abbreviated or full citation lists: \abcite{basic}{others}
\newcommand{\abcite}[2]{\abbr{\cite{#1}}{\cite{#1,#2}}}

% Conference abbreviations: \bibconf[Nth]{SOSP}{Symposium on ...}
\newcommand{\bibconf}[3][]{#1 \bibbrev{#2}{#3 (#2)}}

\newcommand{\ie}{{\em i.e.\xspace}}
\newcommand{\eg}{{\em e.g.\xspace}}

% system related terms
\newcommand{\app}{ConfigV\xspace}

% Fault graph terms

\newcommand{\para}[1]{\smallskip\noindent {\bf #1}}




\makeatletter\if@ACM@journal\makeatother
%% Journal information (used by PACMPL format)
%% Supplied to authors by publisher for camera-ready submission
\acmJournal{PACMPL}
\acmVolume{1}
\acmNumber{1}
\acmArticle{1}
\acmYear{2017}
\acmMonth{1}
\acmDOI{10.1145/nnnnnnn.nnnnnnn}
\startPage{1}
\else\makeatother
%% Conference information (used by SIGPLAN proceedings format)
%% Supplied to authors by publisher for camera-ready submission
\acmConference[ISSTA'17]{ACM SIGSOFT International Symposium on Software Testing and Analysis}{January 01--03, 2017}{New York, NY, USA}
\acmYear{2017}
\acmISBN{978-x-xxxx-xxxx-x/YY/MM}
\acmDOI{10.1145/nnnnnnn.nnnnnnn}
\startPage{1}
\fi


%% Copyright information
%% Supplied to authors (based on authors' rights management selection;
%% see authors.acm.org) by publisher for camera-ready submission
\setcopyright{none}             %% For review submission
%\setcopyright{acmcopyright}
%\setcopyright{acmlicensed}
%\setcopyright{rightsretained}
%\copyrightyear{2017}           %% If different from \acmYear


%% Bibliography style
\bibliographystyle{ACM-Reference-Format}
%% Citation style
%% Note: author/year citations are required for papers published as an
%% issue of PACMPL.
\citestyle{acmauthoryear}   %% For author/year citations



\begin{document}

%% Title information
\title[]{Version Space Learning for Verification on Temporal Differentials}         %% [Short Title] is optional;
                                        %% when present, will be used in
                                        %% header instead of Full Title.
%\titlenote{with title note}             %% \titlenote is optional;
                                        %% can be repeated if necessary;
                                        %% contents suppressed with 'anonymous'
%\subtitle{Subtitle}                     %% \subtitle is optional
%\subtitlenote{with subtitle note}       %% \subtitlenote is optional;
                                        %% can be repeated if necessary;
                                        %% contents suppressed with 'anonymous'


%% Author information
%% Contents and number of authors suppressed with 'anonymous'.
%% Each author should be introduced by \author, followed by
%% \authornote (optional), \orcid (optional), \affiliation, and
%% \email.
%% An author may have multiple affiliations and/or emails; repeat the
%% appropriate command.
%% Many elements are not rendered, but should be provided for metadata
%% extraction tools.

%% Author with single affiliation.
\author{Mark Santolucito}
%\authornote{with author1 note}          %% \authornote is optional;
                                        %% can be repeated if necessary
%\orcid{nnnn-nnnn-nnnn-nnnn}             %% \orcid is optional
\affiliation{
%  \position{Position1}
%  \department{Department1}              %% \department is recommended
  \institution{Yale University}            %% \institution is required
%  \streetaddress{}
  \city{New Haven}
  \state{CT}
%  \postcode{Post-Code1}
  \country{USA}
}
\email{mark.santolucito@yale.edu}          %% \email is recommended

%% Author with two affiliations and emails.
\author{Ruzica Piskac}
%\authornote{with author2 note}          %% \authornote is optional;
                                        %% can be repeated if necessary
%\orcid{nnnn-nnnn-nnnn-nnnn}             %% \orcid is optional
\affiliation{
%  \position{Position1}
%  \department{Department1}              %% \department is recommended
  \institution{Yale University}            %% \institution is required
%  \streetaddress{}
  \city{New Haven}
  \state{CT}
%  \postcode{Post-Code1}
  \country{USA}
}
\email{ruzica.piskac@yale.edu}          %% \email is recommended

%% Paper note
%% The \thanks command may be used to create a "paper note" ---
%% similar to a title note or an author note, but not explicitly
%% associated with a particular element.  It will appear immediately
%% above the permission/copyright statement.
%\thanks{with paper note}                %% \thanks is optional
                                        %% can be repeated if necesary
                                        %% contents suppressed with 'anonymous'


%% Abstract
%% Note: \begin{abstract}...\end{abstract} environment must come
%% before \maketitle command
\begin{abstract}
\begin{abstract}
Rule based classification is an effective machine learning technique that yields low misclassification rates.
However, building a rule based system requires manual creation of large databases of logical constraints.
We present a method to generate rule based systems from temporally structured data.
As an demonstration of this algorithm, we plan to implement a learner that automatically generates constraints for the TravisCI testing framework.
The algorithm will utilize Github commit histories to generate logical constraints that allow us to detect potential build errors without actually building, saving valuable programmer and server time.
\end{abstract}

\end{abstract}


%% 2012 ACM Computing Classification System (CSS) concepts
%% Generate at 'http://dl.acm.org/ccs/ccs.cfm'.
\begin{CCSXML}
%<ccs2012>
%<concept>
%<concept_id>10011007.10011006.10011008</concept_id>
%<concept_desc>Software and its engineering~General programming languages</concept_desc>
%<concept_significance>500</concept_significance>
%</concept>
%<concept>
%<concept_id>10003456.10003457.10003521.10003525</concept_id>
%<concept_desc>Social and professional topics~History of programming languages</concept_desc>
%<concept_significance>300</concept_significance>
%</concept>
%</ccs2012>
\end{CCSXML}

%\ccsdesc[500]{Software and its engineering~General programming languages}
%\ccsdesc[300]{Social and professional topics~History of programming languages}
%% End of generated code


%% Keywords
%% comma separated list
\keywords{Configuration Files, Verification}  %% \keywords is optional


%% \maketitle
%% Note: \maketitle command must come after title commands, author
%% commands, abstract environment, Computing Classification System
%% environment and commands, and keywords command.
\maketitle


\section{Introduction}

%sfddsf~\cite{zhai14heading}


\section{Version Space for Verification}

Version space learning builds a logical constraint model for binary classification, or membership in a set.
ConfigC uses this approach to identify errors in configuration files for MySQL~\cite{Santolucito2016} .
ConfigC learns a model over a set of configuration files that have been labeled as correct, then reports misconfigurations.

Traditional version space learning will use a series disjunctions of a set of predefined hypotheses.
By instead restricting the model to a series of conjunctions, ConfigC can not only flag misconfigurations, but also give the points of failure for non-membership.

To build the model, ConfigC take a single file and derives all possible relations from each file, such as A comes before B, B before C, and A before C.
The user must provide templates for possible relations.
It is assumed if a file is correct, all relations in that file are necessary for another file to be correct ($Correct(C) \implies \forall r \in M(C), r \in Nec$).
In this way, the initial model is built by creating the strongest conditions for a correct file, called the \textit{specific boundary}.
This model is then iteratively relaxed as more examples are seen, a process called \textit{candidate elimination}.
To relax the model, two sets of relations from two files are merged into a single consistent set.

\begin{lstlisting}
n = {}
for (c in files):
  n1 = M(c)
  n = merge(n, n1)
\end{lstlisting}



\subsection{Completeness}

We demonstrate that ConfigC is complete but unsound.
That is, we will always detect a misconfiguration, but may also report correct files as misconfigurations.
We first define the set of incorrect files based on ground truth $Inc_{gt}$, and the set of files which are predicted by the model to be incorrect, $Inc_{prd}$.
To define $Inc_{gt}$, we will introduce the set of relations that are necessary and breaking based on ground truth, $Nec_{gt}$ and $Br_{gt}$ respectively.
The definition of $Inc_{prd}$ follows from the description of the algorithm.
%
\begin{align}
Inc_{gt} = \{&\forall C, \exists r \in Nec_{gt}, r \notin M(C)\\ \nonumber
 &\lor \exists r \in Br_{gt}, r \in M(C)\} \\
Inc_{prd} = \{&\forall C, \exists r \in Nec, r \notin M(C)\}
\end{align}

Our definition of sound and complete are as follows:
%
\begin{align}
&\forall C \in Inc_{gt}, C \in Inc_{prd} \text{  Complete} \\
&\not \exists C \in Inc_{prd}, C \notin Inc_{gt} \text{  Sound}
\end{align}

Completeness follows from showing that $Inc_{gt} \subseteq Inc_{prd}$.
The algorithm is complete if, for any file which breaks a relation for which we have a template (a misconfiguration) and have seen a correct example, the system will mark that file as incorrect.
Since the model always maintains the strongest conditions for correctness for any file, the algorithm is clearly complete.
Completeness corresponds to no false negatives.
However, this algorithm is unsound, as some correct file may be marked as incorrect, i.e. ConfigC returns a false positive.
We have a high false positive rate since the model is an over-approximation of the true model.


\section{Learning from Temporal properties}

%The first step is to build an intermediate representation of the data we will learn.
%This data must be structured as a shallow tree for generalizable learning.
%This restriction is why this approach is not appropriate for language learning on large grammars (such as a programming language).

In ConfigC, the configuration files were analyzed as standalone documents.
Since a TravisCI configuration file is dependent on the code it is trying to build, we must consider a more general sense of configuration file.
We will call this a program summary $P_t$, which is a representation of the repository which contains the information relevant to the learning process.
In the case of TravisCI, this include the \verb|.travis.yml| file, as well as extract key code features that may effect build status, such as programming language and a list of imported libraries.
The summary must contain every piece of information that might lead to a build error.

The subscript on $P_t$ is a time stamp tag based on the ordered commit history.
However, a git history is not a limited to a single linear timeline.
Git features the ability to \textit{branch}, which allows to simultaneous commit chains.
To handle the start of a branch, add a superscript to indicate the branch, and restart the counter on a branch.
To handle the merge of two branches $P_{t}^{x}$ and $P_{t'}^{y}$, step to $P_{t+1}^{x}$, where $x$ is the mainline branch.
We then say that $P_{t'}^{y}$ has no successor commit $P_{t'+1}^{y}$.

We will denote the build status of $P_t$ with $S(P_t)$.
In this application, we consider only the passing and erroring build status, denoted $Pass$, and $Err$ respectively.
All status that are not \verb|error|, as defined by the Travis API, will be included as passing.
For brevity, we denote sequences of build statuses with the following notation:

\begin{align*}
  S(P_t)=Pass \land S(P_{t+1})=Err \implies S(P_{t,t+1}) = PE
\end{align*}

From this summary we can then build a model $M(P_t)$, as in ConfigC, which is the full set of possible relations derivable from the program summary.
In contrast with ConfigC, we now consider both positive and negative examples and so must introduce the \textit{general boundary}.
The general boundary is the dual of the specific boundary, and is the most relaxed requirement for a positive classification.
We denoted specific boundary as the set of necessary relations $Nec$, and now denote the general boundary as the set of breaking relations $Br$.
With this notation, we can formally express the requirement that the program summary is complete.

\begin{align}
  \forall S(P_t)=Err, \exists r \in M(P_t), r \in Br \label{eq:E1}
\end{align}

From the above we know that if a build is erroring, then there must exist at least one error.
By pushing the negation into the formula, we can also know that if a build is passing, then there must not exist any errors.
That is, the model of a passing commit must not contain any rules which are breaking.
Note we are not, however, guaranteed that any rules from a passing commit are necessary.


\begin{align}
  S(P_t) = Err \implies \exists r \in  M (P_t), r \in Br \label{eq:E}\\
  S(P_t) = Pass \implies \forall r \in  M (P_t), r \notin Br \label{eq:P}
\end{align}

While Eq. \ref{eq:E} and \ref{eq:P} might build a basic model, they will do not capture all of the available knowledge.
The key insight is that when we commit a break (P E), we can localize the error to one of the lines that changed.
Either we removed something that was necessary, or added something that was breaking.
We use an inclusive disjunction, since a erroring commit can break multiple things at once.
Expressed formally, where $\setminus$ is the set difference, that is:


\begin{align}
  S(P_{t,t+1}) = PE \implies \nonumber \\
  \exists r \in (M(P_{t})\ \setminus M(P_{t+1})), r \in Nec\ \lor \nonumber \\
  \exists r \in (M(P_{t+1}) \setminus M(P_{t})), r \in Br \label{eq:PE}
\end{align}

We then can combine all these formulas with conjunctions and send it to an SMT solver.
While existential set operations can be expensive on large sets for an SMT solver, in our application this is not the case.
Thanks to the practice of making incremental commits when using source control, these sets will be small and the SMT will be fairly cheap.
In fact, the above implication generalizes to $P_{t,t+n}$, but for efficiency we must require that $M(P_{t})\ \setminus M(P_{t+n})$ is manageably small.
The definition of small here remains to be experimentally determined.


\section{Limitations}

%\subsection{Unsatisfiability}
We have made two assumptions in Eq. \ref{eq:E1}; first that the summary $P_t$ sufficently detailed, i.e. contains every piece of information that might lead to a build error, and second that the model $M(P_t)$ will learn all relations that might lead to a build error, i.e. has a templete for all types of errors.
While these are the strong assumptions - our algorithm is able to detect cases where these assumptions are not met.
If we cannot find a solution for Eq. \ref{eq:PE}, it means either $P_t$ or $M(P_t)$ has been underspecified.
It is then the user's responsibility to expand the definitions accordingly.

\subsection{Trusted Base}
In addition to the completeness of $P_t$ and $M(P_t)$, we must pick a trusted base.
Since it is possible for TravisCI to have bugs, it is possible that between versions of TravisCI, two identical program summaries may have different build statuses.
Since version space learning is (generally) intolerant of noise, we require that
$P_t = P_{t'} \implies S(P_t) = S(P_{t'})$.



\section{Related Work}

\iffalse
Configuration verification has been considered a promising way  
to tackle misconfiguration problems~\cite{xu15systems}.
Nevertheless, a practical and automatic configuration
verification approach still remains an open problem.

\para{Language-support misconfiguration checking}
There have been several language-support efforts proposed for preventing
configuration errors introduced by fundamental deficiencies in
either untyped or low-level languages. For example, in the network
configuration management area, administrators often
produce configuration errors in their routing configuration files.
PRESTO~\cite{enck07configuration} 
automates the generation of device-native configurations
with configlets in a template language. 
Loo {\em et al.}~\cite{loo05declarative} adopt Datalog to reason about 
routing protocols in a declarative fashion. 
COOLAID~\cite{chen10declarative} constructs
a language to describe domain knowledge about devices and
services for convenient network reasoning and management.
Compared with the above efforts, our work focuses on software systems, 
\eg, MySQL and Apache, and our main purpose is to automate configuration
verification rather than proposing new languages 
to convenient configuration-file writing. 

\para{Misconfiguration detection}
Misconfiguration detection techniques aim at checking configuration
efforts before system outages occur.
Most existing detection approaches check 
the configuration files against a set of predefined correctness 
rules, named constraints, and then report errors if 
the checked configuration files do not satisfy these rules.
Huang {\em et al.}~\cite{huang15confvalley},
for example, proposed a 
language, ConfValley, to validate 
whether given configuration files meet administrators' specifications. 
As opposed to our work, ConfValley does not
have inherent misconfiguration checking capability, since it only offers
a language representation and requires administrators to
manually write specifications, which is an error-prone
process. On the contrary, our work does not need users to manually
write anything.

Several machine learning-based misconfiguration detection efforts 
also have been proposed~\cite{yuan11context, zhang14encore, xu16early}.
EnCore~\cite{zhang14encore} introduces a template-based
learning approach to improve the accuracy of their learning results.
The learning process is guided by a set of predefined rule templates
that enforce learning to focus on patterns of interest.
In this way, EnCore filters out irrelevant information and reduces
false positives; moreover, the templates are able to express
system environment information that other machine learning
techniques cannot handle.
Compared with EnCore, \app has the following advantages.
First, \app does not rely on any template. 
Second, EnCore cannot detect missing entry errors, type errors,
ordering errors and fine-grained integer correlation errors,
but \app can detect all of them.
Finally, \app is a very automatic system, but
EnCore needs significant human interventions, \eg, system parameters
and templates.

PCheck~\cite{xu16early} aims to add configuration checking code to the system source code by emulating potential commands and behaviors of the system. 
This emulation is a ``white-box'' approach and requires access to the system's source code.
One drawback to this approach is that for some systems (\eg, ZooKeeper) whose behavior is 
hard to emulate, PCheck cannot automatically generate the corresponding checking code.
Due to the emulation based testing strategy, PCheck's scope is limited to reliability problems caused by misconfiguration parameters. 
In contrast, \app is a ``black-box'' approach and only requires a training set of configuration files to learn rules.
By using a rule learning strategy of examples, \app is able to detect general misconfiguration issues that are outside the scope of emulation testing (\eg memory or thread usage settings), including performance, security, availability and reliability.

\para{Misconfiguration diagnosis}
Misconfiguration diagnosis approaches have been proposed to address configuration problems post-mortem.
For example, ConfAid~\cite{attariyan10automating} 
and X-ray~\cite{attariyan12x-ray} use dynamic information
flow tracking to find possible configuration errors that may have resulted in
failures or performance problems. AutoBash~\cite{su07autobash} 
tracks causality and automatically fixes 
misconfigurations. Unlike \app, most misconfiguration
diagnosis efforts aim at finding errors after system
failures occur, which leads to prolonged recovery time.

\fi




%% Acknowledgments
\begin{acks}                            %% acks environment is optional
                                        %% contents suppressed with 'anonymous'
  %% Commands \grantsponsor{<sponsorID>}{<name>}{<url>} and
  %% \grantnum[<url>]{<sponsorID>}{<number>} should be used to
  %% acknowledge financial support and will be used by metadata
  %% extraction tools.
  This material is based upon work supported by the
  \grantsponsor{GS100000001}{National Science
    Foundation}{http://dx.doi.org/10.13039/100000001} under Grant
  No.~\grantnum{GS100000001}{nnnnnnn} and Grant
  No.~\grantnum{GS100000001}{mmmmmmm}.  Any opinions, findings, and
  conclusions or recommendations expressed in this material are those
  of the author and do not necessarily reflect the views of the
  National Science Foundation.
\end{acks}


%% Bibliography
\bibliography{biblio}


\newpage

%% Appendix
%\appendix


\end{document}
