
\subsection{Completeness}

We demonstrate that ConfigC is complete but unsound.
That is, we will always detect a misconfiguration, but may also report correct files as misconfigurations.
We first define the set of incorrect files based on ground truth $Inc_{gt}$, and the set of files which are predicted by the model to be incorrect, $Inc_{prd}$.
To define $Inc_{gt}$, we will introduce the set of relations that are necessary and breaking based on ground truth, $Nec_{gt}$ and $Br_{gt}$ respectively.
The definition of $Inc_{prd}$ follows from the description of the algorithm.
%
\begin{align}
Inc_{gt} = \{&\forall C, \exists r \in Nec_{gt}, r \notin M(C)\\ \nonumber
 &\lor \exists r \in Br_{gt}, r \in M(C)\} \\
Inc_{prd} = \{&\forall C, \exists r \in Nec, r \notin M(C)\}
\end{align}

Our definition of sound and complete are as follows:
%
\begin{align}
&\forall C \in Inc_{gt}, C \in Inc_{prd} \text{  Complete} \\
&\not \exists C \in Inc_{prd}, C \notin Inc_{gt} \text{  Sound}
\end{align}

Completeness follows from showing that $Inc_{gt} \subseteq Inc_{prd}$.
The algorithm is complete if, for any file which breaks a relation for which we have a template (a misconfiguration) and have seen a correct example, the system will mark that file as incorrect.
Since the model always maintains the strongest conditions for correctness for any file, the algorithm is clearly complete.
Completeness corresponds to no false negatives.
However, this algorithm is unsound, as some correct file may be marked as incorrect, i.e. ConfigC returns a false positive.
We have a high false positive rate since the model is an over-approximation of the true model.
