\section{Properties of Temporal Learning}

We have made two assumptions in Eq. \ref{eq:E1}; first that the summary $P_t$ sufficiently detailed, i.e. contains every piece of information that might lead to a build error, and second that the model $M(P_t)$ will learn all relations that might lead to a build error, i.e. has a template for all types of errors.
While these are strong assumptions, our algorithm is able to detect cases where these conditions are not met.
If we cannot find a solution for Eq. \ref{eq:PE}, it means either $P_t$ or $M(P_t)$ has been underspecified and is insufficiently detailed, and the definitions must be expanded accordingly.

In addition to the sufficient detail of $P_t$ and $M(P_t)$, we must pick a trusted base.
Since TravisCI may not always be backwards compatible, it is possible that between versions of TravisCI, two identical program summaries may have different build statuses.
Since version space learning is (generally) intolerant of noise, we require that
$P_t = P_{t'} \implies S(P_t) = S(P_{t'})$.
If we allow for noisy data, for example in the case that it is not possible to identify a trusted base, this predicate will no longer hold.
This situation would require a noise tolerant version space learning~\cite{hong1997generalized}, though this may degrade the completeness guarantees.

Picking the right assumptions upon which to build a completeness and soundness proof remain future work, either with a trusted base, or in the noisy context.
Choosing the right assumption and building these proofs are a valuable direction, as it will allow us to identify which parts of the model and training set are most important to ensure coverage of the system.

