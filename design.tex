
\section{\app Design}

\xxx{
\app Design: 
  \begin{itemize}
  \item Present an architecture of \app with a fig. Briefly describe 
    how it works (step by step).
  \item Detail learner part
  \item Merge
  \item Check 
  \item Limitations
  \end{itemize}
}


We are not quite machine learning because we use more formal reasoning.
This let's us have a nice garuntee that while we have false positives, but not false negatives.

\forall errors, error \in report
not \forall report r \in error


Another benefit is that, unlike machine learning algorithms, such as nueral nets, we can actually see the source of errors and the systems reasoning about why a line should be considered an error.
In nueral nets, we just get a boolean value, and the mechanics of the system are entirly lost. 
The language model is then recoverable and editable.

The system is also flexible so that changes to the algorithm can be made at a high level.


A rule R is added to the set of all rules,
  if $\exists$ learning file f s.t. R(f) is non vacuously true
That rule R is then removed from the set of all rules,
  if $\exists$ learning file f st R(f) is false.

This is accomplished in two passes.
First we collect all possible rules for every file.
Then we merge the all the rules to create our final set.
This conviently gives rise to an embarresingly parallel situation, which Haskell allows us to easily take advantage of by
  using the parallel mapping library parmap.

\begin{lstlisting}
  potentialRules = parmap findAllRules learningSet.
  finalRules = foldl1 mergeRules potenialRules
\end{lstlisting}


Each rule is of the Attribute typeclass, which means a rule must support the following operations:
\begin{lstlisting}
class Attribute r where
  learn :: ConfigFile Common -> [r]
  merge :: [r] -> [r] -> [r] 
  check :: [r] -> ConfigFile Common -> Error
\end{lstlisting} 


\subsection{learn}
  For a single given file, we take very line ordering to be a rule.
\subsection{merge}
  Then when merging these sets of rules, we take the intersection of the rules infered on the individual files.
  Maybe we could also do something like only taking rules that show up multiple times.
\subsection{check}
  To check a file by using a rule set, we simply take all the rules that are releveant to the user's file.
  Rules that are relavent are the ones where both parts of the ordering are present.
  We learn the rule set for the user file, and every rule in the learned set must be present in the user file.
