\begin{abstract}

Software failures resulting from configuration errors have been
commonplaces as modern software systems become increasingly larger
and more complex. Although existing efforts aim to overcome
misconfiguration problems by either building post-failure error diagnosis 
tools or hardening systems based on misconfiguration reports, these
retroactive solutions, nevertheless, require human (\eg, user) 
intervention, thus leading to impractical administration cost. 
%especially in large-scale software systems. 
Complementing existing post-failure forensics, 
this paper proposes, \app, a {\em proactive} misconfiguration checker.
With \app in hand, the users of software systems can submit their
configuration files to \app for misconfiguration detection
{\em before} deploying them in the real software settings.
At the heart of \app lies a machine learning mechanism, 
which learns rules and constraints from pluggable 
correct sampling configuration files, 
and then detects errors violating the learned constraints.
\app mainly offers three significant benefits.
First, there is no any extra burden added to users who only need to 
simply use \app as a ``seamless'' pre-checker for their configurations.
Second, because many misconfiguration errors have been eliminated 
by \app, the workloads of post-failure forensics in runtime
are significantly reduced, thus making these tools truly practical.
Finally, since there have been many correct but not specific 
example configuration files in practice, 
increasingly more samples can be freely added to \app,
thus giving \app evolutionary capability.

\end{abstract}
