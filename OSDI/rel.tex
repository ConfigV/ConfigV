
\section{Related Work}

Providing language support has been considered as a promising means
of tackling configuration problems~\cite{xu15systems}.
Nevertheless, practical language-based misconfiguration
detection approach still remains an open problem.

\para{Configuration languages.}
There have been several language-support efforts proposed to prevent
configuration errors introduced by fundamental deficiencies in
either untyped or low-level languages. For example, in network
configuration management area, it is easy for administrators to
produce configuration errors in their routing configuration files.
PRESTO~\cite{} automates the generation of device-native configurations
with configlets in a template language. 
Loo {\em et al.}~\cite{} adopt Datalog to reason about 
routing protocols in a declarative fashion. COOLAID~\cite{} constructs
a language to describe domain knowledge about network devices and
services to convenient network reasoning and management.

Compared with these existing efforts, 
our work mainly focused on software systems, \eg, MySQL and Apache,
rather than network configurations. In addition, we do not need 
the user of \app to manually write a configuration file with the proposed
language, since \app can automatically parse a target configuration
file into our proposed representation.

\para{Misconfiguration detection.}
Misconfiguration detection techniques aim at checking the configuration
efforts before the system outages occur.
Most of existing detection approaches check 
the configuration files against a set of predefined correctness 
rules, named constraints, and then report the errors if 
the checked configuration files do not satisfy these rules.

Huang {\em et al.}~\cite{huang15confvalley} proposed a specification 
language, ConfValley, to validate 
whether given configuration files meet administrators' 
``belief'' in mind. Different from \app, ConfValley itself does not
have inherent misconfiguration checking capability, since it only offers
a language representation. In addition, administrators have to
manually write specifications with ConfValley, which is an error-prone
process; on the contrary, \app does not need users to manually
write anything.

Several machine learning-based misconfiguration detection efforts 
also have been proposed~\cite{yuan11context, zhang14encore}.
Specifically, EnCore~\cite{zhang14encore} introduces a template-based
learning approach to improve the accuracy of their learning results.
The learning process is guided by a set of predefined rule templates
that enforce learning to focus on patterns of interests.
By this way, EnCore filters out irrelevant information and reduces
the false positives; moreover, the templates are able to express
system environment information that state-of-the-art machine learning
techniques cannot handle.
Compared with EnCore, \app has the following advantages.
First, \app does not rely on whether the files in given configuration set 
are 100\% correct. Second, \app can cover much more types of 
misconfigurations, since \app introduces probabilistic type checking
mechanism. Finally, \app is a language framework, which could 
even be used to write configuration files, but EnCore is only a 
misconfiguration detection tool.

\para{Misconfiguration diagnosis.}
Many misconfiguration diagnosis approaches have been proposed~\cite{}.
For example, ConfAid~\cite{} and X-ray~\cite{} use dynamic information
flow tracking to find possible configuration errors that may result in
failures or performance problems. AutoBash~\cite{} speculatively
executes processes and tracks causality to automatically fix 
misconfigurations. Different from \app, most of misconfiguration
diagnosis efforts aim at finding out errors after the system
failures occur, which typically lead to prolonged recover time.

\para{Misconfiguration tolerance.}
There have been several efforts proposed to test whether systems are 
tolerant to misconfigurations~\cite{xu13do}. 
ConfErr~\cite{} uses a human error hodel from psychology and
linguistics to inject misconfigurations into systems.
SPEX~\cite{xu13do} takes a white-box testing approach to automatically
extract configuration parameter constraints from source code and generates 
misconfigurations to test whether systems can tolerate the potential
configuration errors.

Making systems gracefully handle misconfigurations and eliminating
configuration errors are two orthogonal directions.
The former helps improve the robustness of systems and make 
diagnosis easier. This is especially important for 
software that will be widely distributed to end users.
Our work belongs to the latter case, which is used to 
prevent configuration errors before the system failures occur.
