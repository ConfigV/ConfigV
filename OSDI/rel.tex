
\section{Related Work}

Language support has been considered a promising way  
to tackle configuration problems~\cite{xu15systems}.
Nevertheless, a practical language-based misconfiguration
detection approach still remains an open problem.

\para{Configuration languages.}
There have been several language support efforts proposed for preventing
configuration errors introduced by fundamental deficiencies in
either untyped or low-level languages. For example, in the network
configuration management area, it is easy for administrators to
produce configuration errors in their routing configuration files.
PRESTO~\cite{enck07configuration} 
automates the generation of device-native configurations
with configlets in a template language. 
Loo {\em et al.}~\cite{loo05declarative} adopt Datalog to reason about 
routing protocols in a declarative fashion. 
COOLAID~\cite{chen10declarative} constructs
a language to describe domain knowledge about devices and
services for convenient network reasoning and management.

Compared with these existing efforts, 
our work mainly focused on software systems, \eg, MySQL and Apache,
rather than network configurations. In addition, we do not need 
the user of \app to manually write a configuration file with the proposed
language, since \app can automatically parse a target configuration
file into our proposed representation.

\para{Misconfiguration detection.}
Misconfiguration detection techniques aim at checking configuration
efforts before system outages occur.
Most existing detection approaches check 
the configuration files against a set of predefined correctness 
rules, named constraints, and then report errors if 
the checked configuration files do not satisfy these rules.

Huang {\em et al.}~\cite{huang15confvalley} proposed a 
language, ConfValley, to validate 
whether given configuration files meet administrators' specifications. 
Different from \app, ConfValley does not
have inherent misconfiguration checking capability, since it only offers
a language representation and requires administrators to
manually write specifications, which is an error-prone
process. On the contrary, \app does not need users to manually
write anything.

Several machine learning-based misconfiguration detection efforts 
also have been proposed~\cite{yuan11context, zhang14encore}.
EnCore~\cite{zhang14encore} is the work closest to \app.
It introduces a template-based
learning approach to improve the accuracy of their learning results.
The learning process is guided by a set of predefined rule templates
that enforce learning to focus on patterns of interest.
In this way, EnCore filters out irrelevant information and reduces
false positives; moreover, the templates are able to express
system environment information that other machine learning
techniques cannot handle.
Compared to EnCore, \app has the following advantages.
Firstly, \app does not rely on 100\% correctness in the files of the given configuration set. 
Secondly, \app not only covers many more types of 
misconfigurations, but also introduces probabilistic types.
Finally, \app is a language framework, which can 
even be used to write configuration files, but EnCore is only a 
misconfiguration detection tool.

\para{Misconfiguration diagnosis.}
Many misconfiguration diagnosis approaches have been proposed%
~\cite{attariyan10automating, attariyan12x-ray}.
For example, ConfAid~\cite{attariyan10automating} 
and X-ray~\cite{attariyan12x-ray} use dynamic information
flow tracking to find possible configuration errors that may result in
failures or performance problems. AutoBash~\cite{su07autobash} 
speculatively
executes processes and tracks causality to automatically fix 
misconfigurations. Unlike \app, most misconfiguration
diagnosis efforts aim at finding errors after system
failures occur, which typically leads to prolonged recovery time.

\para{Misconfiguration tolerance.}
There have been several efforts proposed to test whether systems are 
tolerant to misconfigurations~\cite{xu13do}. 
%ConfErr~\cite{} uses a human error hodel from psychology and
%linguistics to inject misconfigurations into systems.
SPEX~\cite{xu13do} takes a white-box testing approach to automatically
extract configuration parameter constraints from source code and generates 
misconfigurations to test whether systems can tolerate potential
configuration errors.

Making systems gracefully handle misconfigurations and eliminating
configuration errors are two orthogonal directions.
The former helps improve the robustness of systems and make 
diagnosis easier. This is especially important for 
software that will be widely distributed to end users.
Our work belongs to the latter case, which is used to 
prevent configuration errors before system failures occur.
