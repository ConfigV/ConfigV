
\section{Learner}
\label{sec-learn}

The goal of learner module is to derive rules and constraints from
the intermediate representation generated by the translator.
In general, the learner module has two parts.
The first part is to learn rules to check configuration errors like
missing entry, ordering errors, and fine-grained value correlation errors. 
These errors tend to cause total system failures.
%Once the configuration file has been validated against such rules, 
%the user may choose to invoke a more sensitive constraint checker. 
The second part of the learner aims to derive 
constraints on entries to check for suspicious (\ie, anomalous) values 
that may violate standard practice (like the example~4 in
$\S$\ref{sec-motiv}).


\subsection{Derivation of Probabilistic Rules}
\label{subsec-rules}

\para{A strawman solution.}
The first step of \app is to learn relations 
that must hold over multiple parts of a configuration file.
We now first present a strawman solution (employed by
previous work~\cite{santolucitoCAV, zhang14encore}) that uses 
a set of correct configuration files as a learning set, 
from which it was possible to derive rules 
that must hold with absolute certainty. 
However, this is approach is severely limiting. 
It is difficult to obtain a set of files 
that is both guaranteed to be without misconfiguration 
and large enough to learn the many rules of configuration files.
This usually requires manual verification of the learning set, 
which is prone to error.

As a result of this restriction, 
these efforts only consider a rule if it holds over exactly every file in 
the learning set. This behavior can be formally described as follows:


\begin{small}
\begin{flalign*}
C =&\ \text{Correct Learning Set}\\
\text{::}\ & \text{\{Configuration Files in Intermediate Representation\}}\\
LR =&\ \text{Learned Rules :: \{Rule\}}\\
RR =&\ \text{Reported Rules :: \{Rule\}}\\ 
LR =&\ \{ r\ \mid \forall file \in C,\ holds(r,file)\} \\
RR =&\ \{ r\ \mid r \in L\ \land \neg\ holds(r,userfile) \}
\end{flalign*}
\end{small}

Each rule can be thought of as a mapping from 
lines $j$ and $k$ in a configuration file to a relation, $R$.

\[
\{ Rule = (a_j, a_k) | j \neq k \} \rightarrow \{ R \}
\]

Specifically, $a_j$ and $a_k$ are two different lines from our
intermediate representation, or more formally, $j \neq k \land
a_j, a_k \in \{ L \}$ where is the set of lines (key-value pairs) found
in the learning set. The relation $R$
is a Boolean function specific to the error to detect. 
As an example, to detect the error that key $k_1$ must always 
have a value greater than key $k_2$, the relation $R$ is $>$.

\para{Our approach.}
In \app, the rule learning mechanism is tolerant 
enough to accept a dataset {\em full of} incorrect configuration files.
Rather than manually correcting each file, 
%we observe that the files are usually incorrect in several lines.
We extend the previous formalism to run probabilistic learning
on our intermediate representations (generated by the translator). 

Our probabilistic approaches for learning the missing values, 
ordering, and fine-grained value correlation rules stem 
from our previous work with building 
the non-probabilistic versions of these rule-learning algorithms. 
We start with the idea that for each of these rules, 
we are going to consider all possible pairs of keys that appear in every 
file, and for our learning process, calculate the likelihood that each of 
these pairs constitute a rule. 
In essence, this is a mechanism that will consider many different possible 
pairs of lines in the file, 
and attempt to compile a set of these pairs 
that are expressed more often than others as a basis for finding patterns 
within the example set that can be used to evaluate new files.

More formally, this approach can be defined as follows. 
The output of our learning algorithm is augmented to a map from 
key-value pairs to a probability distribution over number of possible 
relations.

\[
\{ P\_Rule = (a_j, a_k) | j \neq k \} \rightarrow \{ (R_1, R_2, ... , R_n) \}
\]

We can think of each of these $(a_j, a_k)$ as possibly having a different relationship, defined by the set $\{ R_i \}$, which cover the entire outcome space of possible relationships between the two values $(a_j, a_k)$.

For the entry missing rules, we define $R_1$ as the event that $a_j$ and
$a_k$ appear together, and $R_2$ to be the event that $a_j$ appears
without $a_k$, or by the transitive equivalent, $a_k$ appears without
$a_j$. For the ordering rules, we define $R_1$ as the event that
$a_j$ appears before $a_k$ and $R_2$ be the event that $a_k$ appears
before $a_j$. For the value correlation rules, we define $R_1$ as the
event that $a_j \leq a_k$, $R_2$ the event that $a_j = a_k$, and $R_3$
the case that $a_j \geq a_k$. Notice that the $R_i$ do not have to be
disjoint, but only have to union to the entire probability space.

By examining the learning set, we will derive a distribution of the set $\{R_i\}$ based on how many times we observe an occurrence of each relation. This distribution will then be used at checking time to determine if a user's configuration has broken a likely rule. 

\begin{small}
\begin{flalign*}
I\ \ =&\ \text{Incorrect Learning Set}\\
\text{::}\ & \text{\{Configuration Files in Intermediate Representation\}}\\
LP =&\ \text{Learned Probabilistic Rules :: \{(P\_Rule)\}}\\
RP =&\ \text{Reported Probabilistic Rules :: \{(P\_Rule)\}}\\
LP =&\ \text{count\_relation\_occurrences}(I)\\
RP =&\ \{ r\ \mid r \in LP\ \land \Pi(r)>p \land \neg r(userfile) \}
\end{flalign*}
\end{small}

A rule will be reported as broken if the probability the rule is correct, $\Pi$, is greater than some user defined constant, $p$. This constant can be adjust to the user's preference. A small $p$ will increase the likelihood of finding an error, but also increase the number of false positives that are reported.


\subsection{Learning Suspicious Constraints}
\label{subsec-constraints}

With a configuration file that has been verified against catastrophic
failures, the user may also use \app to find more subtle issues.
Anomalous values can cause tricky, but impactful, performance and memory
issues that are hard to debug, as discussed in Example 4 of 
$\S$\ref{sec-motiv}. 
Consequently, suspicious values should be flagged and a warning returned
to the user indicating the violation.

We now describe the technique we use to detect anomalous values for 
numerical attributes. Let $A$ be the set of attributes contained in the 
configuration files in the sample dataset. 
Let $A_n$ be the subset of attributes of $A$ which are numerically typed. 
Then, for each attribute $a \in A_n$, we construct a vector $v_a$ of the 
values corresponding to attribute $a$, seen over the entire sample dataset.
For each $v_a$, we compute 
an interval  $$[\hat{v_a} - 50*MAD(v_a), \hat{v_a} + 50*MAD(v_a)],$$ 
where $\hat{v_a}$ represents the median over the values 
in $v_a$ and $MAD(v_a$) refers to the 
median absolute deviation. 
This is a variant of a standard outlier detection test, namely the Hampel identifier.\footnote{Mathematically, $MAD(v_a) = 1.4826* median(|v_a - \hat{v_a}|)$, estimating standard deviation 
for a normal distribution.} 
In the checking phase, as long as the checker finds a value for a numerical 
attribute in the checked file outside of this interval, 
a warning would be printed to the user indicating the violating value, 
the attribute, and the upper or lower Hampel threshold. 

The intuition behind this is that if the user has input a value 
that falls outside of an interval containing values that are considered 
``normal'' over the entire sample dataset, 
that value will probably cause an error, in particular for performance. 
We cannot know for sure if this value will cause an issue. 
For instance, a user might have a machine with 
particularly high-end hardware, 
in which case a value beyond the upper Hampel threshold may be appropriate. 
