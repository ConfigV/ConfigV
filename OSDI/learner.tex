
\section{Learner}
\label{sec-learn}

\ennan{This section details how do we extract rules from the
intermediate representations. This section may should have the following
two subsections: Aaron's part (including Xinyu's fine-grained
value correlation rule learning) and Jonathon's part.
In addition, we may need to add one more subsection about
learning system environment information related rules,
\eg, for paths and users.}

In general, the learner module has two parts of learning tasks.
One is to learn rules in order to check configuration errors like
entry missing, ordering errors, and correlation errors.
Another one is to learn constraints to check suspicious (or called
anomalous) singular values that may violate standard values.

\subsection{Learning Rules}
\label{subsec-rules}

\subsection{Learning Suspicious Constraints}
\label{subsec-constraints}
Anomalous values can cause tricky performance and memory issues that are hard to debug,
as discussed in Example 4 of Section \ref{sec-motiv}. Consequently, suspicious values 
should be flagged and a warning returned to the user indicating the violation.

We now describe the technique we use to detect anomalous values for 
numerical attributes. Let $A$ be the set of attributes contained in the 
configuration files in the sample dataset. 
Let $A_n$ be the subset of attributes of $A$ which are numerically typed. 
Then, for each attribute $a \in A_n$, we construct a vector $v_a$ of the 
values corresponding to attribute $a$, seen over the entire sample dataset.
For each $v_a$, we compute 
an interval  $$[\hat{v_a} - 3*MAD(v_a), \hat{v_a} + 3*MAD(v_a)],$$ 
where $\hat{v_a}$ represents the median over the values 
in $v_a$ and $MAD(v_a$) refers to the 
median absolute deviation. 
This is a standard outlier detection test, namely the Hampel identifier. 
\footnote{Mathematically, $MAD(v_a) = 1.4826* median(|v_a - \hat{v_a}|)$, estimating standard deviation 
for a normal distribution.} 
In the checking phase, as long as the checker finds a value for a numerical 
attribute in the checked file outside of this interval, 
a warning would be printed to the user indicating the violating value, 
the attribute, and the upper or lower Hampel threshold. 

The intuition behind this is that if the user has input a value 
that falls outside of an interval containing values that are considered 
``normal'' over the entire sample dataset, 
that value will probably cause an error, in particular for performance. 
We cannot know for sure if this value will cause an issue. 
For instance, a user might have a machine with 
particularly high-end hardware, 
in which case a value beyond the upper Hampel threshold may be appropriate. 
