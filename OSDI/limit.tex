
\section{Discussion and Limitations}

This section discusses a few \app's limitations
and possible solutions.

\para{Legal misconfigurations.}
While \app can check diverse configuration errors without
human participations, all the misconfiguration detection techniques,
including \app, cannot handle some types of configuration errors
resulting from events occurred during system runtime.
Such configuration errors are referred to as {\em legal misconfigurations}%
~\cite{yin11anempirical}. As shown in Yin {\em et al.}%
~\cite{yin11anempirical}, many parameter misconfigurations have 
perfectly legal parameters but do not deliver the functionality intended
by users. These cases are more difficult to detect by
automatic checkers and may require more user training or
better configuration design.
A potential solution is to combine existing misconfiguration diagnosis
tools, \eg, X-ray~\cite{attariyan12x-ray}, 
with \app in order to enhance the 
misconfiguration checking capability.

\para{Misconfiguration across software components.}
As exposed by Yin {\em et al.}~\cite{yin11anempirical},
cross-software configuration correlation problems also account
for a considerable number of misconfiguration cases.
For example, in a LAMP-based Web server, one entry in 
PHP configuration file, {\tt mysql.max\_persistent = 400}
may make users encounter a ``too many connections'' error,
because a correlated entry in the underlying MySQL's configuration
file assigns {\tt max\_connection} to 300, which is less
than the MySQL connection numbers in PHP's configuration file (\ie, 400).
It is quite difficult to detect such a type of tricky error
through leaning approaches, because not only users or engineers 
are not aware of the hidden interactions~\cite{xu15systems},
but also it is hard to obtain a global knowledge to the entire
configurations.
One possible solution to this problem might be to introduce
some cryptographic protocol, \eg, private set
intersection~\cite{kissner05privacy}, to privately extract the
overlapping entries, \eg, {\tt mysql.max\_connection} in the 
above MySQL and PHP case, for double-checking.

\para{Network configuration verification.}
\app mainly focuses on software configurations, \eg, MySQL and Apache,
so that our approach is limited to support network configuration
verification. This is because network configurations have quite
different representations, format and rules from software configurations,
since network configurations are typically written in 
more domain-specific policy languages.
In fact, many network verification tools, 
\eg, NoD~\cite{lopes15checking} and 
Dobrescu {\em et al.}~\cite{dobrescu14software},
have been proposed to check whether network configurations
meet their specifications.

\ennan{One more limitation description here.} 
