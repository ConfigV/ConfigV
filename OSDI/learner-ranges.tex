
\section{Ranges}
\label{sec-learn-ranges}

We discuss the technique we use to detect anomalous values for numerical attributes. For a given test file, let $A$ be the set of attributes contained in the file. Let $A_n$ be the subset of attributes of $A$ which are numerical types. Then, for each attribute $a \in A_n$, we construct a vector $v_a$ of the values corresponding to attribute $a$ seen over the entire data-set. For each $v_a$, we compute an interval  $$[\hat{v_a} - 3*MAD(v_a), \hat{v_a} + 3*MAD(v_a)],$$ 
where $\hat{v_a}$ represents the median over the values in $v_a$ and $MAD(v_a$) refers to the median absolute deviation. This is a standard outlier detection test, namely the Hampel identifier. \footnote{Mathematically, $MAD(v_a) = 1.4826* median(|v_a - \hat{v_a}|)$, estimating standard deviation for a normal distribution.} If the value for a numerical attribute in a test file falls outside of this interval, a warning is printed to the user indicating the violating value, the attribute, and the upper or lower Hampel threshold. 

