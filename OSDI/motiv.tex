\section{Motivating Examples}
\label{sec-motiv}

In this section, we present several tricky configuration errors
extracted from real-world examples. 
Existing efforts, \eg, learning-based checking
efforts~\cite{zhang14encore, wang04automatic}, cannot detect
these sophisticated misconfiguration problems.

\para{Example~1: Ordering errors.} 
The first example configuration error is extracted from 
Yin {\em et al.}~\cite{yin11anempirical}.
When a user configure PHP to run with the
Apache HTTP Server, this user writes, among others, the following lines
in the configuration file.

\begin{lstlisting}[language=C]
extension = mysql.so
...
extension = recode.so
\end{lstlisting} 

Such a configuration file will cause the Apache server to 
fail to start due to a segmentation fault error~\cite{yin11anempirical}. 
This is because when using PHP in Apache, the extension {\tt mysql.so} 
depends on {\tt recode.so}, and the relative ordering of two of them 
is crucial. \app is able to inform the user that {\tt recode.so} 
should appear before {\tt mysql.so}, and reports the error.

\begin{lstlisting}[language=C]
ORDERING ERROR: Expected "extension" "recode.so"
BEFORE "extension" "mysql.so"
\end{lstlisting} 

\para{Example~2: Fine-grained value correlation errors.} 
In a real-world misconfiguration example~\cite{correlation}, 
a user configures his MySQL as the following:

\begin{lstlisting}[language=C]
key_buffer_size = 384M
max_heap_table_size = 128M
max_connections = 64
thread_cache_size = 8
...
sort_buffer_size = 32M
join_buffer_size = 32M
read_buffer_size = 32M
read_rnd_buffer_size = 8M
...
\end{lstlisting} 

This user complained his MySQL's load is very high and the website's
respond speed is very slow.
In this case, {\tt key\_buffer\_size} is used by all the threads
cooperatively while {\tt join\_buffer} and {\tt sort\_buffer} are 
created by each thread for private use; thus, the maximum amount
of used key buffer, \ie, {\tt key\_buffer\_size} should be larger than 
{\tt join|sort\_buffer\_size} * {\tt max\_connections}. 
Clearly, in the above example, it is not, so that this misconfiguration
causes MySQL loads very slow.

\app is able to check such fine-grained value correlation problem.
If we run \app on this configuration file, \app  would return:

\begin{lstlisting}[language=C]
INTEGER RELATION ERROR:
Expected "key_buffer_size">="max_connections"*"sort_buffer_size"
\end{lstlisting} 

Note that existing effort, EnCore, aims to detect value correlation 
errors. However, different from \app, EnCore can only detect simple
correlation, \eg, {\tt mysql.max\_persistent} < {\tt max\_connections},
rather than the fine-grained computation-based value correlation
done by \app.

\para{Example~3: Entry missing errors.} 
If a user wants to use OpenLDAP to enable her directory access
protocol, she needs to use the password policy overlay. This is usually
done through the following entries in the OpenLDAP configuration file:

\begin{lstlisting}[language=C]
include schema/ppolicy.schema
overlay ppolicy
\end{lstlisting} 

When using the password policy overlay in OpenLDAP, 
we have to first include the related schema.
Leaving out the ``include'' statement will cause the failure of 
this LDAP. Running \app on such a misconfiguration file would return:

\begin{lstlisting}[language=C]
MISSING KEYWORD ERROR: Expected "overlay" "ppolicy"
in the same file: "include" "schema/ppolicy.schema"
\end{lstlisting} 

\ennan{Example~4: This should be the singular value error.}

\para{Other errors.}
\app can also deal with the configuration errors that can be detected by
existing work, such as EnCore~\cite{zhang14encore}.
For example, both \app and EnCore can handle system execution related
misconfiguration problems.
Furthermore, \app is also able to detect type errors and syntax errors
in configuration files.
