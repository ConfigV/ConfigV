\section{Motivating Examples}
\label{sec-motiv}

In this section we illustrate functionality of \app on several non-trivial configuration errors
extracted from real-world examples. Although the errors are relatively simple, we call them 
non-trivial, because the majority of existing tools, \eg, learning-based checking
tools~\cite{zhang14encore, wang04automatic}, cannot detect
these configuration errors. Most of the presented examples were found on StackOverflow,
a popular question and answer website for programmers. To better understand problems that users have with 
configuration files, we explored and analyzed misconfigurations on a large number of user forums and on-line discussion sites.

\para{Example~1: Ordering error.} 
Ordering errors were reported by Yin {\em et al.}~\cite{yin11anempirical} and our first
example illustrates how ordering errors can cause a system to crash. When a user configures PHP 
to run with the
Apache HTTP Server, most likely the user will take some already existing configuration files and adapt them
to suit her needs. The configuration file might contain, among others, 
the following lines:
\begin{lstlisting}[language=C, xleftmargin=.01\textwidth]
extension = mysql.so
...
extension = recode.so
\end{lstlisting} 

This configuration file will cause the Apache server to 
fail to start due to a segmentation fault error. 
This is because, when using PHP in Apache, the extension {\tt mysql.so} 
depends on {\tt recode.so}, and their relative ordering
is crucial. 
We call the above example of a misconfiguration file
an {\em ordering error}.
Yin {\em et al.} report that ordering errors widely exist in
many system configurations, \eg, PHP and MySQL,
and typically lead to multiple system crash events.
However, no existing tool can effectively solve 
or detect this problem~\cite{zhang14encore, xu15systems, xu13do}.

By invoking \app, the user can detect such a configuration error.
In particular, \app reports that {\tt recode.so} 
should appear before {\tt mysql.so}, as shown
below:

\begin{lstlisting}[language=C, xleftmargin=.01\textwidth]
ORDERING ERROR: Expected "extension" "recode.so"
BEFORE "extension" "mysql.so"
\end{lstlisting} 

\para{Example~2: Fine-grained value correlation error.} 
Our next real-world misconfiguration example~\cite{correlation} comes from a discussion on StackOverflow.
The user has configured her MySQL as in the following:

\begin{lstlisting}[language=C, xleftmargin=.01\textwidth]
key_buffer_size = 384M
max_heap_table_size = 128M
max_connections = 64
thread_cache_size = 8
...
sort_buffer_size = 32M
join_buffer_size = 32M
read_buffer_size = 32M
read_rnd_buffer_size = 8M
...
\end{lstlisting} 

The user complains that her MySQL load was very high, causing the website's
response speed to be very slow.
In this case, {\tt key\_buffer\_size} is used by all the threads
cooperatively, while {\tt join\_buffer} and {\tt sort\_buffer} are 
created by each thread for private use; thus, the maximum amount
of used key buffer, \ie, {\tt key\_buffer\_size}, should be larger than 
{\tt join|sort\_buffer\_size} * {\tt max\_connections}. 
Clearly, in the above example, it does not hold, 
so this misconfiguration causes MySQL to load very slowly.

If we run \app on this configuration file, \app  would return:

\begin{lstlisting}[language=C, xleftmargin=.01\textwidth]
INTEGER RELATION ERROR:
Expected "key_buffer_size" >= "max_connections" * "sort_buffer_size"
\end{lstlisting} 

We can see that during the learning process not only do we learn simple statements that compare two values, 
but also we learn more complex correlations. We call these more complex relations fine-grained value correlations, and the errors
\emph{fine-grained value correlation errors}. 
This type of error is more sophisticated than the simple value correlation that some tools can detect~\cite{yin11anempirical, zhang14encore}.
A typical value correlation error
states that one entry's value should have a certain correlation with
another entry's value. For example, in MySQL,
the value of {\tt max\_connections} should be higher than
{\tt mysql.max\_persistent}. 

Our tool can learn this simple correlation as well, but learning more complex properties requires a different 
approach to the learning 
process. We need to track several variables. It is not enough just to compare them, but we need to learn, as in this particular case, which
invariant must be preserved. It was pointed out by Xu {\em et al.}~\cite{xu15hey} that
detecting fine-grained value correlation errors 
is a much more challenging
task than the normal value correlation problem.

To the best of our knowledge, \app is the first tool that
is able to check such fine-grained value correlation problems.


\para{Example~3: Missing entry error.} 
Many critical system outages result from the fact that an important
entry was missing in the configuration file. 
We call such a problem a {\em missing entry error}.
In a public misconfiguration dataset~\cite{configdataset},
many MySQL failure reports were caused by
missing entry errors.
Below is a real-world entry missing error example~\cite{yin11anempirical}:
when a user wants to use OpenLDAP to enable her directory access
protocol, she needs to use the password policy overlay. This is usually
achieved via the following entries in the OpenLDAP configuration file:

\begin{lstlisting}[language=C, xleftmargin=.01\textwidth]
include schema/ppolicy.schema
overlay ppolicy
\end{lstlisting} 

When using the password policy overlay in OpenLDAP, 
users must first include the related schema.
Leaving out the {\tt include schema/ppolicy.schema} entry, 
as done by many users~\cite{yin11anempirical}, 
causes the failure of LDAP. 
If the user runs \app on such a misconfiguration file,
\app would return:

\begin{lstlisting}[language=C, xleftmargin=.01\textwidth]
MISSING KEYWORD ERROR: Expected "overlay" "ppolicy"
in the same file: "include" "schema/ppolicy.schema"
\end{lstlisting} 

\para{Example~4: Singular value anomalies.} 
Many system performance problems are caused by the
anomaly that a value is set either too high or too low.
For example, a parameter relating to memory might be too large,
exhausting the RAM and causing extreme slowness or even a crash. 
Consider the following real-world misconfiguration file%
~\cite{singleValue}: a user was experiencing the ``My SQL Server 
has gone away" error.  
This is difficult to debug, since the error message is not specific.
It turns out the following line in her configuration file was problematic:

\begin{lstlisting}[language=C, xleftmargin=.01\textwidth]
max_allowed_packet = 100M
\end{lstlisting} 

The user eventually resolved the issue by replacing the above line with {\tt max\_allowed\_packet=2M}.
One possible way to reach this conclusion more quickly is to determine that the original value, 100 MB, 
is statistically deviant and extremely large for the {\tt max\_allowed\_packet} attribute. 
ConfigV can detect these
outliers. Run on the user's original configuration file, ConfigV would output:

\begin{lstlisting}[language=C, xleftmargin=.01\textwidth]
WARNING: Violated Upper Hampel Rule for max_allowed_packet with value 104857600 
\end{lstlisting} 

In the above, 104857600 represents the integer value of 100 megabytes. The idea is to output
a warning to the user that her value falls outside of a range considered normal, so that she
can make the appropriate adjustment to the value herself. In the above situation, it is 
prudent to set the value of {\tt max\_allowed\_packet} lower. We call such a value a 
\emph{singular value anomaly}.

\para{Other errors.}
In addition to the given motivating examples \app can also detect and report configuration errors that can be exposed by
existing work, such as EnCore~\cite{zhang14encore} and
CODE~\cite{yuan11context} (a full scope of \app's capabilities is given in $\S$\ref{sec-eval}).
A recently reported configuration error made the MySQL
daemon fail to start~\cite{syserror}.
One entry was written as 
{\tt datadir=/root/appfinder/mysql}, and the type, as well as 
format, seemed correct.
However, the problem is that this directory should not 
contain the root directory; the correct entry
should be {\tt datadir=/appfinder/mysql}.
Such an error is called a system environment-related configuration error.
Both \app and EnCore can handle misconfigurations concerning system execution.
Furthermore, \app is also able to detect type errors and syntax errors
in configuration files. An example of a type error is assigning to a variable {\tt {general\_log}} 
a file path {\tt {/var/log/mysql/mysql.log}}. \app successfully reports this typing error: 
\begin{lstlisting}[language=C, xleftmargin=.01\textwidth]
TYPE ERROR: Expected a Int with P=1.0 for "general_log[mysqld]"
\end{lstlisting}
