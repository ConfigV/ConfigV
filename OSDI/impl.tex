\section{Implementation}

\app is primarily implemented in Haskell, and the suspicious values
detection module was written in R. 
%The suspicious value module reimplements the
%main functionality of the Haskell system, and is is a testament to the
%simplicity of the algorithm and the ease of reproduction. 
For clarity,
we only present the key components of the Haskell implementation here.

The translator is developed as a parser for the grammar
described in $\S$\ref{sec-trans}. 
%The implementation presented no interesting challenges. 

The learning and checking modules are developed to allow
for customized extensibility. In order to verify a configuration file
against a new type of error, a user only needs to provide a new type
that is an instance of the \textit{Attribute} typeclass (\ie, define some
functions over that type). 
In particular, for each type of error the users
wish to detect, they must implement three functions;
\lstinline{learn}, \lstinline{merge}, and \lstinline{check}.

\com{
\begin{lstlisting}
class Foldable t => RuleSet t a where
  learn :: IRConfigFile -> t a
  merge :: t a -> t a -> t a
  check :: t a -> IRConfigFile -> [Error]
\end{lstlisting}

This typeclass is polymorphic over the data structure as denoted by 
\lstinline{Foldable t =>}, 
which means user can choose any data structure they prefer.
In general, the best data structure is a hashmap 
from key-value pairs to relations, 
previously defined as a \textit{P\_Rule} in $\S$\ref{subsec-rules}.
}

The \lstinline{learn} function takes the intermediate representation of 
a single configuration file (as generated by the translator), 
and generates the set of rules that can be derived from that file.
Specifically, these rules take the form of 
a \textit{P\_Rule} to track the counts of relation events.
The \lstinline{merge} function will unify two sets of rules, from two separate configuration files, into a single consistent set of rules. 
For the probabilistic rules, this tends to be simply a union between the two sets. 
When two equivalent rules are merged, the relation event counts should be summed to update the probability distributions.

The \lstinline{check} function will start all the rules we have learned, 
and filter until we only have the rules that have been broken by the file 
of interest. We first eliminate the rules 
that do not pass the probability check defined in $\S$\ref{subsec-rules}.
Second, we only consider the rules which are relevant to the file -- we 
do not need to check rules about the entries 
that do not appear in the file.
Lastly, we take the errors 
as any rule that does not hold over the given file.


\com{
A sketch of a typical implementation of these three functions is shown below. Again, the specifics of this implementation will vary based on the type of error being detected. 

\begin{lstlisting}
learn f = 
  count_relation_events (all_line_pairs f)
merge s1 s2 = 
  unionWith sum s1 s2
check rs f = 
  likely_rs  = filter prob_check rs
  related_rs = filter applies_to_file likely_rs
  errors_rs  = filter (is_valid f) related_rs
 in errors
\end{lstlisting}
}



\begin{table*}[tbp]
\centering
\caption{Sampled benchmarks for misconfiguration detection}
\label{table-casestudy}
\begin{small}
\begin{tabular}{|l|l|l|l|}
\hline
{\bf ID} & {\bf Problem Description} & {\bf Error Type} & 
{\bf \app Report}  \\ 
\hline
\hline
1 & MySQL cannot create test file  
& Entry Missing 
& Expected innodb\_force\_recovery=1 in the \\ & & 
&  same file as: innodb\_strict\_mode=1\\ \hline

2 & How can my wait\_timeout/interactive\_timeout  
& Entry Missing
& Expected set\_time=1 in the same\\ & be ignored 
& & file as wait\_timeout \\ \hline

3 & MySQL on EC2 instance becomes very slow
& Correlation
& Expected key\_buffer\_size $>=$ \\ & after the first query
&& thread\_size * sort\_buffer\_size \\ \hline

4 & MySQL cannot successfully make  
& Type Error
& Expected master-host[mysqld] should 
 \\ & replication &&  be a valid IP address\\ \hline

5 & MySQL running on production servers has
& Type Error
& Expected set-variable[myisamchk] \\ &  gone away error in memory
&&  should be a memory size like 20M \\ \hline

6 & Cannot start MySQL 5.5 as normal 
& Ordering
& Expected socket[mysqld]=/var/lib/ \\ & user Fedora 15 
&&  mysql/mysql.sock BEFORE  \\ & && user[mysqld]=mysql

\\ \hline

7 & Fail to login MySQL
& Type Error
& Expected old\_passwords[mysqld] 
 \\ &&&  should be a valid value \\ \hline

8 & MySQL running at CentOS cannot know 
& Ordering
& Expected localhost before bind-
 \\ & enable named pipe
&&  address[mysqld] \\ \hline

9 & Troubles installing MySQL5 via  Darwin 
& Ordering
& Expected port[mysqld]=3306 BEFORE \\ & Ports
&&  socket[mysqld]=/tmp/mysql.sock \\ \hline

10 & Fail to install Percona xtraDB 
& Entry Missing
& Expected wsrep\_node\_address[mysqld] \\ & cluster on ubuntu 13.04
&&  in the same file as: wsrep\_sst\_method \\ &&& [mysqld] \\ \hline

11 & MySQL access denied for user rootlocalhost 
& Ordering
& Expected port[mysqld]=3306  \\ & &&   BEFORE socket[mysqld]= \\ &&& /tmp/mysql.sock
\\ \hline

12 & MySQL Partition Problem
& Correlation
& Expected innodb\_flush\_max  \\ & &&  \_commit[mysqld] $>=$ innodb \\ &&&  \_support[mysqld] \\ \hline

13 & mysql\_upgrade script problems on 
& Type Error
& Expected log-slow-queries[mysqld]  \\ & MySQL 5.0.24 
&&  should be a path type\\ \hline


14 & MySQL: bug report!!
& Ordering
& Expected innodb\_file\_io\_threads[mysqld]  \\ & 
&& BEFORE innodb\_log\_files\_in\_group[mysqld]\\ \hline

15 & MySQL max connections changed 
& Correlation
& Expected innodb\_file\_per\_table \\ & without notice
&&  [mysqld] $<=$ max\_connections[mysqld]  \\ \hline

16 & Enabling log in MySQL 5 6 
& Type Error
& Expected sql\_mode[mysqld] should  \\ & prevents server from starting
&&  be a string \\ \hline

17 & MySQL needs help for rapidly growing 
& Correlation
& Expected innodb\_buffer\_pool\_size \\ & table and decreasing speed
&&  [mysqld] $>$ sort\_buffer\_size[mysqld] \\ \hline

18 & MySQL filled all RAM on the system
& Correlation
& Expected join\_buffer\_szie * \\ & && max\_connections $<=$ key\_buffer\_size \\ \hline


19 & MySQL hit a max limit
& Correlation
& Expected join\_buffer\_size * \\ &&& max\_connections $<=$ key\_buffer\_size \\ \hline

20 & Cannot create a new thread
& Correlation
& Expected join\_buffer\_size * \\ &&& max\_connections $<=$ key\_buffer\_size \\ \hline

\end{tabular}
\end{small}
\end{table*}




%The rest of the system is structural code to handle these modules for the user. 
The core learning algorithm will learn over each file individually, 
then merge the results of learning together. 
Since we learn each set of rules on each file in isolation from the others, we have a pleasingly parallel situation.
The \lstinline{learn} function can then be called in parallel over the entire learning set.
As a functional language, there is no programming overhead to implement
this in Haskell -- we simply import the parallel mapping library \cite{parallel} to replace all uses of  \lstinline{map} with \lstinline{parmap}.
The merge stage could also be parallelized by using a divide and conquer approach, but this is not a priority since the learning stage only needs to be run once per learning set, and can then be cached (in our case, as a .json file) for fast reading when doing verification.

The source code for our implementation is available at {\em (URL omitted for blind review)}.
