
\section*{Abstract}

System failures resulting from configuration errors 
have become the major causes affecting the availability and
reliability of today's software systems.
Although many misconfiguration handling techniques,
\eg, misconfiguration checking, troubleshooting and repair, 
have been proposed, 
offering automatic verification for configuration files -- like
what we did to regular programs -- is still an open problem.
This is because software configurations are typically written in
poorly structured and untyped ``languages'', and 
specifying constraints and rules for configuration 
verification is non-trivial in practice.

This paper presents, \app, the first automatic verification framework for
general software configurations.
\app verifies a target configuration file $F$ through three steps.
First, \app analyzes a dataset containing many sample configuration 
files belonging to the same system as $F$,
and translates these sample files to a
well-structured and probabilistically-typed 
intermediate representation.
Second, \app derives rules and constraints by analyzing
this intermediate representation, thus building a
language model.
Finally, \app uses the resulting language model to verify $F$.
\app framework is highly modular, 
does not rely on the system source code or templates and
can be applied to any new configuration file type with minimal user input. 

\app is capable of detecting various tricky errors that cannot
be detected by previous efforts,
including ordering errors, fine-grained value correlation errors, 
entry missing errors, and environment related errors. 
%Our tool, named \app, relies on an abstract representation of language 
%rules to allow for this modularity. 
\ennan{One or two sentences here describe the most important
experimental results. For example, we verify real-world misconfiguration 
files and find 95\% errors are detected.}
