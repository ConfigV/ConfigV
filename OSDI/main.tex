
\documentclass[letterpaper,twocolumn,10pt]{article}

\usepackage{times}            % standard fixed width font
\usepackage{usenix}
\usepackage{algorithm}
\usepackage{algpseudocode}
%\usepackage{algorithmic}

\usepackage{graphicx}
\usepackage{amsmath}
\usepackage{xspace}
\usepackage{footnote}
\usepackage{cite}
\usepackage{amsfonts}
\usepackage{subfig}
%\usepackage{natbib}
\usepackage{hhline}
\usepackage{multirow}
\usepackage{setspace} 
\usepackage{epsfig}
\usepackage[hyphens]{url}
\usepackage[colorlinks,linkcolor=blue,citecolor=blue,urlcolor=blue]{hyperref}
\usepackage[hyphenbreaks]{breakurl}
\usepackage{booktabs}
\usepackage[compact]{titlesec}
%\usepackage{xcolor}
%\usepackage[algoruled,vlined,ruled,linesnumbered]{algorithm2e}
\usepackage{lipsum}
\usepackage{listings}


\lstset{
  basicstyle=\footnotesize\ttfamily,
  breaklines=true,
  frame=bottomline,
  language=haskell,
  %identifierstyle=\color{identifierColor},
  backgroundcolor=\color{white},   % choose the background color; you must add \usepackage{color} or \usepackage{xcolor}
  breakatwhitespace=false,         % sets if automatic breaks should only happen at whitespace
  %captionpos=b,                    % sets the caption-position to bottom
  %commentstyle=\color{mygreen},    % comment style
  %frame=single,	                   % adds a frame around the code
  keepspaces=true,                 % keeps spaces in text, useful for keeping indentation of code (possibly needs columns=flexible)
  keywordstyle=\color{blue},       % keyword style
  otherkeywords={*,let, Server, Replication, FaultGraph, rankRCG, print, fialProb, goal, ...},           % if you want to add more keywords to the set
  numbers=left,                    % where to put the line-numbers; possible values are (none, left, right)
  numbersep=5pt,                   % how far the line-numbers are from the code
  rulecolor=\color{black},         % if not set, the frame-color may be changed on line-breaks within not-black text (e.g. comments (green here))
  showtabs=false,                  % show tabs within strings adding particular underscores
  stepnumber=1,                    % the step between two line-numbers. If it's 1, each line will be numbered
  %title=\lstname                  % show the filename of files included with \lstinputlisting; also try caption instead of title
  mathescape=true,
  tabsize=3,
  literate=*{->}{{\textcolor{blue}{$\to$}}}{1}
           {<-}{{\textcolor{blue}{$\leftarrow$}}}{1}
}

\usepackage[T1]{fontenc}
%Ruzica\usepackage[scaled=0.78]{DejaVuSansMono}


\clubpenalty=10000      % penalty for creating a club line at end of line.
\widowpenalty=10000     % penalty for creating a widow line at top of page.

% Select one or other if want to see comments.
% \com is sometimes displayed during draft.
\long\def\com#1{}
%\long\def\com#1{{\bf \sc comment: }{\small [#1]}{\bf \sc\ endcomment}\newline}



\long\def\ennan#1{{\color{blue}{\bf Ennan: }{\small [#1]}}}
\long\def\markk#1{{\color{violet}{\bf Mark: }{\small [#1]}}}
\long\def\ruzica#1{{\color{red}{\bf Ruzica: }{\small [#1]}}}
\long\def\jon#1{{\color{black}{\bf Jon: }{\small [#1]}}}
%\long\def\xxx#1{}

% Use this macro to force page breaks where ugly widows/orphans occur;
% be sure to recheck all uses after any significant change to the text!
\def\widowpage{\pagebreak}

% Choose abbreviated or long-version alternatives in paper
%\long\def\abbr#1#2{#1}			% abbreviated version
\long\def\abbr#1#2{#2}			% long version

% Choose abbreviations or long names/titles in bibliography
%\def\bibbrev#1#2{#1}			% short version
%\def\bibbrev#1#2{#2}			% long version
\def\bibbrev#1#2{\abbr{#1}{#2}}		% follow abbr macro

% Abbreviated or full citation lists: \abcite{basic}{others}
\newcommand{\abcite}[2]{\abbr{\cite{#1}}{\cite{#1,#2}}}

% Conference abbreviations: \bibconf[Nth]{SOSP}{Symposium on ...}
\newcommand{\bibconf}[3][]{#1 \bibbrev{#2}{#3 (#2)}}

\newcommand{\ie}{{\em i.e.\xspace}}
\newcommand{\eg}{{\em e.g.\xspace}}

% system related terms
\newcommand{\app}{ConfigV\xspace}

% Fault graph terms

\newcommand{\para}[1]{\smallskip\noindent {\bf #1}}



\begin{document}



\special{papersize=8.5in,11in}
\setlength{\pdfpageheight}{\paperheight}
\setlength{\pdfpagewidth}{\paperwidth}

% Uncomment one of the following two, if you are not going for the 
% traditional copyright transfer agreement.

%\exclusivelicense                % ACM gets exclusive license to publish, 
                                  % you retain copyright

%\permissiontopublish             % ACM gets nonexclusive license to publish
                                  % (paid open-access papers, 
                                  % short abstracts)

%\titlebanner{banner above paper title}        % These are ignored unless
%\preprintfooter{short description of paper}   % 'preprint' option specified.

\title{An Automatic Verification Framework for Software Configurations}

\author{~}

\maketitle
%\vspace{-50pt}


\begin{abstract}

System failures resulting from configuration errors 
are one of the major reasons for compromised reliability of today's software
systems. Although many techniques have been proposed for a 
configuration error
detection, these approaches mainly can be applied after an error has occurred. Verifying configuration files is, nevertheless, 
a challenging problem,
because 1) software configurations are typically written in
poorly structured and untyped ``languages'', and 
2) specifying rules for configuration 
verification is challenging in practice.
This paper presents \app, the first automatic verification framework for
general software configurations.
Our framework works as follows: in the pre-processing stage, we first automatically derive a specification.
Once we have a specification, we check if a given configuration file adheres to that specification.
The process of learning specification works through three steps.
First, \app parses a training set of configuration 
files (not necessarily all correct) into a
well-structured and probabilistically-typed 
intermediate representation.
Second, based on the association rule
learning algorithm \app learns rules from 
these intermediate representations. These rules are 
establishing relationships between the keywords appearing in the files. 
Finally, \app employs rule graph analysis to refine the 
resulting rules. \app is capable of detecting various configuration errors,
including ordering errors, integer correlation errors, type errors,
and missing entry errors. We evaluated \app by verifying 
public configuration files on Github, and we show that \app can 
detect known configuration errors in these files.

\end{abstract}


\section{Introduction}

Machine learning has been used in various software analysis techniques, such as programming-by-example \cite{lau2000version}, invariant synthesis \cite{garg2014ice}, and error detection \cite{Santolucito2016}.
However, many popular machine learning algorithms, such as neural nets and n-gram models, are not designed to provide simple justifications for their classification results.
While effective in practice, the lack of justification for the results limits the applicability to software analysis, where justification is often critical.
For example, in the case of error detection, the system should not only report which files have errors, but also locate the errors.
Version space learning is a technique for logical constraint classification~\cite{mitchell82}.
As a logical constraint system, version space learning can provide the justification that is difficult to obtain from other methods, but its main weakness is that it cannot handle noisy data.

%However, many popular machine learning algorithms, such as neural nets and n-gram models, produce probabilistic models of correctness.
%While effective in practice, these approachs cannot be garunteed to be complete, that is they will always find the error.
%We use version space learning to address the need for a formal completness garuntee in automated model generation for verification.

In our previous work~\cite{Santolucito2016} we built a prototype to verify configuration files from a set of correct examples.
The tool, ConfigC, takes as a training set a set of correct configuration files and builds a set of rules describing a language model.
This model is then used to check if a new configuration file adhears to those rules.
If a file is incorrect, the tool identifies the location of error, and reports what is incorrect.
For example, ConfigC might output \texttt{Ordering Error: Expected "extension mysql.so" before "extension recode.so"}.
While this prototype's results are promising, the main weakness are that the learning can only use correct configuration files.
Additionally, there is a relatively high false positive rate (marking correct files as incorrect).

In order to extend the prototype, in this paper we provide a new description of ConfigC in terms of version space learning.
This description will allow us to understand how the justifications are produced in ConfigC.
We can then extend ConfigC with a new algorithm to decrease the false positive rate, while maintaining the ability to provide justification.
%and show that the results are garunteed to be complete.

%Although the results are complete, the approach used previously produces a high false positive rate.
We propose an extension to ConfigC that allows it to handle both correct and incorrect files in the training set.
Additionally, we extend ConfigC with the ability to use structure within the training set.
While the previous training set from ConfigC was a random sampling of configuration files, many other training sets have a structure based on version history.
%The algorithm we propose takes advantage of this commonly availible, but underutilized structure of training sets for software analysis.
%Because most code does not exist in isolation, but changes over time with development, any training set of code from a verison control system, like Github, has a rich temporal structure. 
This structure is simply a partial order, and can be used by our proposed algorithm for more effective learning.
%A learning algorithm that uses the temporal structure of code to create sound classification with a low false positive rate can be used in at least the above listed software analysis techniques.

To test this approach in practice, we plan to implement our algorithm to check for TravisCI\footnote{http://www.travis-ci.com} configuration errors.
TravisCI is a continuous integration tool connected to Github that allows programmers to automatically run their test suite on every code update (commit).
A user adds a configuration file to the repository that enables TravisCI and specifies build conditions, such as which compiler to use, which dependencies are required, and a set of benchmarks to test.
This ensures the tool can always be automatically built correctly on a fresh machine.

A recent usage study~\cite{API} of TravisCI found that 15-20\% of failed TravisCI builds are due to "errors" - which is the TravisCI code used to mean the configuration file was malformed and the software could not even be built.
Using the data from \cite{API}, we can also learn that since the start of 2014, approximately 88,000 hours of server time was used on TravisCI projects that resulted in an error status.
This number not only represents lost server time, but also lost developer time, as programmers must wait to verify that their work does not break the build.
If these malformed projects could be quickly statically checked on the client side, both TravisCI and its users could benefit.

Our main contributions are then as follow:

\begin{enumerate}

\item We give a description of our prototype tool, ConfigC, in the context of version space learning.
\item We propose a new algorithm for ConfigC to handle both incorrect and correct training data, as well as internal structure of a training set.
\item We propose a real-world application of this approach, and outline the steps needed for an implementation.

\end{enumerate} 

\section{Motivating Examples}
\label{sec-motiv}

In this section we present the capability of \app through 
detecting errors in several real-world misconfiguration examples. 
These examples were non-trivial configuration errors
that were reported on StackOverflow~\cite{stackoverflow},
a popular question and answer website for programmers and administrators. 
%To better understand these problems, 
%we explored and analyzed misconfigurations 
%on a large number of user forums and on-line discussion sites.

\para{Example~1: Ordering error.} 
Ordering errors were reported by Yin {\em et al.}~\cite{yin11anempirical} and our first
example illustrates how ordering errors can cause a system to crash. When a user configures PHP 
to run with the
Apache HTTP Server, most likely the user will take some already existing configuration files and adapt them
to suit her needs. The configuration file might contain, among others, 
the following lines:

\begin{lstlisting}[language=C, xleftmargin=.01\textwidth]
    extension = mysql.so
        ...
    extension = recode.so
\end{lstlisting}

This configuration file will cause the Apache server to 
fail to start due to a segmentation fault error. 
This is because, when using PHP in Apache, the extension {\tt mysql.so} 
depends on {\tt recode.so}, and their relative ordering
is crucial. 
We call the above example of a misconfiguration file
an {\em ordering error}.
Yin {\em et al.} report that ordering errors widely exist in
many system configurations, \eg, PHP and MySQL,
and typically lead to multiple system crash events.
However, no existing tool can effectively solve 
or detect this problem~\cite{zhang14encore, xu15systems, xu13do}.

By invoking \app, the user can detect such a configuration error.
In particular, \app reports that {\tt recode.so} 
should appear before {\tt mysql.so}, as shown
below: \ennan{Mark, please update all the motivating examples'
output results according
to our newest version implementation.}

\begin{lstlisting}[language=C, xleftmargin=.01\textwidth]
    ORDERING ERROR: Expected "extension" "recode.so"
    BEFORE "extension" "mysql.so"
\end{lstlisting} 

\para{Example~2: Fine-grained value correlation error.} 
Our second misconfiguration example~\cite{correlation} 
comes from a discussion on StackOverflow.
The user has configured her MySQL as in the following:

\begin{lstlisting}[language=C, xleftmargin=.01\textwidth]
    key_buffer_size = 384M
    max_heap_table_size = 128M
    max_connections = 64
    thread_cache_size = 8
        ...
    sort_buffer_size = 32M
    join_buffer_size = 32M
    read_buffer_size = 32M
    read_rnd_buffer_size = 8M
        ...
\end{lstlisting} 

The user complains that her MySQL load was very high, 
causing the website's
response speed to be very slow.
In this case, {\tt key\_buffer\_size} is used by all the threads
cooperatively, while {\tt join\_buffer} and {\tt sort\_buffer} are 
created by each thread for private use; thus, the maximum amount
of used key buffer, \ie, {\tt key\_buffer\_size}, should be larger than 
{\tt join|sort\_buffer\_size} * {\tt max\_connections}. 
Clearly, in the above example, it does not hold, 
so this misconfiguration causes MySQL to load very slowly.

If we run \app on this configuration file, \app  would return:

\begin{lstlisting}[language=C, xleftmargin=.01\textwidth]
    INTEGER RELATION ERROR:
    Expected "key_buffer_size" >= "max_connections" * "sort_buffer_size"
\end{lstlisting} 

We can see that during the learning process not only do we learn simple statements that compare two values, 
but also we learn more complex correlations. We call these more complex relations fine-grained value correlations, and the errors
\emph{fine-grained value correlation errors}. 
This type of error is more sophisticated than the simple value correlation that some tools can detect~\cite{yin11anempirical, zhang14encore}.
A typical value correlation error
states that one entry's value should have a certain correlation with
another entry's value. For example, in MySQL,
the value of {\tt max\_connections} should be higher than
{\tt mysql.max\_persistent}. 

Our tool can learn this simple correlation as well, but learning more complex properties requires a different 
approach to the learning 
process. We need to track several variables. It is not enough just to compare them, but we need to learn, as in this particular case, which
invariant must be preserved. It was pointed out by Xu {\em et al.}~\cite{xu15hey} that
detecting fine-grained value correlation errors 
is a much more challenging
task than the normal value correlation problem.
To the best of our knowledge, \app is the first tool that
is able to check such fine-grained value correlation problems.

\para{Example~3: Missing entry error.} 
Many critical system outages result from the fact that an important
entry was missing from the configuration file. 
We call such a problem a {\em missing entry error}.
In a public misconfiguration dataset~\cite{configdataset},
many MySQL failure reports were caused by
missing entry errors.
Below is a real-world missing entry error example~\cite{yin11anempirical}:
when a user wants to use OpenLDAP to enable her directory access
protocol, she needs to use the password policy overlay. This is usually
achieved via the following entries in the OpenLDAP configuration file:

\begin{lstlisting}[language=C, xleftmargin=.01\textwidth]
    include schema/ppolicy.schema
    overlay ppolicy
\end{lstlisting} 

When using the password policy overlay in OpenLDAP, 
users must first include the related schema.
Leaving out the {\tt include schema/ppolicy.schema} entry, 
as done by many users~\cite{yin11anempirical}, 
causes the failure of LDAP. 
If the user runs \app on such a misconfiguration file,
\app would return:

\begin{lstlisting}[language=C, xleftmargin=.01\textwidth]
    MISSING KEYWORD ERROR: Expected "overlay" "ppolicy"
    in the same file: "include" "schema/ppolicy.schema"
\end{lstlisting} 

\para{Example~4: Type errors.} 
Many system performance problems are caused by 
assigning incorrect type of values to some key in configuration
files. Consider the following real-world misconfiguration file: 
a user tries to install MySQL and she needs to initiate the path
of the log information generated by MySQL.
This user puts the following assignment in her MySQL
configuration file: 

\begin{lstlisting}[language=C, xleftmargin=.01\textwidth]
    general_log = /var/log/mysql/mysql.log
\end{lstlisting} 

Unbeknowest to this user, the entry ``general\_log'' should be an 
integer, not a string. In MySQL, there is another entry named
``general\_log\_file'' used to specify the log path.
With \app, this user can get the following result:

\begin{lstlisting}[language=C, xleftmargin=.01\textwidth]
    TYPE ERROR: Expected a Int with P=1.0 for
    "general_log [mysql]"
\end{lstlisting} 

%The above result means that we need to assign an integer value to
%the entry ``general\_log''. 

%\section{Learning the Rules}
\label{sec:system}

Before describing how our system for learning rules works, we first outline a translator which converts configuration files into an intermediary representation. Typing is based on an
already given set of basic types. However, typing is also a system module than can be easily extended to support
more types. In that case the user will need to provide rules for type inference and probability distributions for values where type inference is ambiguous.

\subsection{From Untyped Expressions to Quantum Types}

Configuration files usually consist of variable assignments, or some simple instructions about including files or libraries. We initially
collect all assignments into a set of pairs $(n, v)$, where $n$ denotes 
variable name and $v$ denotes variable value. However, based only on the 
value $v$ we cannot determine the type of $n$.

Consider for instance the following example:\\
\texttt{\hspace*{2em}foo = 300\\
\hspace*{2em}bar = 300.txt}\\
Most likely \texttt{foo} is an integer, but it could also be a string.
In the latter case we want to learn the rule $ \texttt{foo} \in \textsf{substrings}(\texttt{bar})$. We introduce {\emph{quantum types}} to 
address this issue.

Let $\mathcal{T}$ be a set of basic types. In \app set $\mathcal{T}$ contains strings, integers, file paths, sizes and IP addresses. 
A quantum type built from $\mathcal{T}$ is a list of pairs $[(\tau_1, p_1),\ldots,(\tau_n, p_n)]$ such that $\tau_i \in \mathcal{T}$, 
$0 \le p_i \le 1$ 
and $\Sigma p_i = 1$. These probabilities are learned from the training 
corpus.

When a value has a quantum type, we generate rules for all its types. This means that by assigning {\texttt{foo}} a quantum type 
(e.g. $(\texttt{foo}, 300, [(\textsl{Int},90\%),(\textsl{String},10\%)])$
we now generate rules for strings and integers.

For a given value, its quantum type is updated over time. When the type inference uniquely determines the type, the probability of all other types is set to zero, and  the associated rules are withdrawn.

This idea is closely related to existentially quantified types \cite{Launchbury93lazyfunctional}.



\section{The \app Framework Overview}

\begin{figure*}[tbp] \centering
\includegraphics[width=0.84\textwidth]{figs/overview}
\caption{\app's workflow. The green components represent configuration 
  files, including both training set of configuration files and users' input
  configuration files to verify.
  The blue dashed box is \app. 
  The yellow components are key modules of \app.
  The purple circle is a checker for checking whether the target
  configuration file violates any rule.}
\label{fig-overview}
\end{figure*}

We propose \app, an automatic verification framework for 
software configuration files.
As shown in Sec.~\ref{sec-motiv}, \app can detect many sophisticated 
configuration errors, including ordering errors, missing entry errors,
type errors and integer correlation errors. 
Figure~\ref{fig-overview} shows
a typical \app verification workflow with three steps:
translation, learning, and rule refinement. This section briefly
describes how each step works.

\para{Initial phase}
We start with the assumption 
that we are given a number of (not necessarily correct) 
configuration files, called {\em training set}, 
belonging to the same system, such as MySQL or Apache. 
These files, therefore, follow similar patterns.
%In the following steps,
%we will exploit in a collection of learning algorithms 
%to build rules that describe a language model for the files.

\para{Translator}
The translator module first parses the input training  
set of configuration files, and then transforms them into 
a more structured and typed intermediate representation.
When we infer the types of entries in a configuration file, 
the type of an entry cannot always be fully determined from 
a single value, since it is difficult to understand
the purposes of key-value entries in modern
software configuration files~\cite{xu15hey}.
We address this problem 
by introducing {\em probabilistic types}.
Rather than giving a variable a single type, 
we assign several types over a probability distribution. 
We can later use these well-structured files
as a ``true'' training set to learn the rules. 

\para{Learning}
The learner reads a set of configuration  files that have been translated
into well-structured representations. 
The learner employs a learning algorithm, {\em rule association 
algorithm}, to generate various rules,
potentially used to handle different types of configuration errors,
including ordering errors, missing entry errors,
type errors and integer correlation errors.
These rules are the outputs of the learner, 
and will be used to detect errors later.
Because the translator gives the learner probabilistically typed entries,
the learner is also responsible for determining types for these entries.

\para{Post Analysis}
Finally, the logically structured representation of learned rules allows for a further post analysis stage.
This novel extension infers knowledge on the accepted rules that can improve the output of the system.
As a case study, we build a graph to model the learned rules.
We analyze the properties of this graph to construct an ordering of rules by their importance, 
  as well as to produce a measure of complexity for any configuration of the target system.
While the metrics in used in \app are effective, they are not intended to be exhaustive.
The information contained in the structured representation of the learned rules, 
  is a unique benefit of the learning algorithm, as contrasted with other machine learning techniques.



\section{Translator}
\label{sec:trans}

The translator takes as input a training set of configuration files and transforms it into a typed and well-structured intermediate representation.
The translator can be seen a parser used to generate an intermediate representation for the learner module (cf. Sec.~\ref{sec-learn}).
Translating or parsing is system dependent since each configuration language (MySQL, Apache, PHP) uses a different grammar.
\app allows users to provide extra help to the translator for their specific system configurations.

The translator converts each key-value assignment $k=v$ in the configuration file to a triple $(k, v, \tau)$, where $\tau$ is the type of $v$. 
There are two major challenges in this step.
First is that configuration files' keywords are not necessarily unique and may have some additional context (modules or conditionals).
To solve this, we rely on the fact that keywords in a configuration file must be unique within their context, and rename all keywords with their context.
The set of unique keys, $\keys$, for the sample training set in Figure~\ref{fig:tset} would then be [``{\tt foo[server]}'',``{\tt bar[client]}''].

{
\setlength{\belowcaptionskip}{-15pt}
\begin{figure}[!htb]
    \centering
    \begin{minipage}{.25\textwidth}
	\begin{lstlisting} [label={lst:file1.cnf},language=C,caption={file1.cnf}]
[server]
foo = ON
[client]
bar = 1
	\end{lstlisting}
    \end{minipage}%
    \hspace{1cm}
    \begin{minipage}{0.25\textwidth}
	\begin{lstlisting} [label={lst:file2.cnf},language=C,caption={file2.cnf}]
[server]
foo = ON
[client]
bar = ON
	\end{lstlisting}
    \end{minipage}
    \hspace{1cm}
    \begin{minipage}{0.25\textwidth}
	\begin{lstlisting} [label={lst:file3.cnf},language=C,caption={file3.cnf}]
[server]
foo = OFF
[client]
bar = OFF
	\end{lstlisting}
    \end{minipage}
    \caption{A sample training set of configuration files}
    \label{fig:tset}
\end{figure}
}

\para{Probabilistic Types}
%\label{sec:ptypes}
An additional challenge is that it is not always possible to fully determine the type of key based on a single example value. 
For this reason, we introduce \textit{probabilistic types}, as contrasted with \textit{basic types}.
In \app, the set of basic types contains strings, file paths, integers, sizes, and Booleans. 
Taking the configuration \ref{lst:file1.cnf}, we can assume {\tt foo} is a Boolean type by the grammar of MySQL,
  but the keyword {\tt bar} could many types.
If we choose the type on the first example it will be a integer type, if we choose a type that fits all examples it will be a string, or we might somehow select the Boolean type.
Let us assume that there is a critical rule we must learn that the Boolean keywords should have the same values, $eq(\texttt{foo},\texttt{bar})$.
If we take {\tt bar}$::int$, we do not learn the above rule, nor do we learn this rule with {\tt bar}$::string$ - only with {\tt bar}$::bool$ is the rule is valid.
To resolve this ambiguity, and choose the best type, the translator assigns a distribution of types to a keyword based on examples from the training set (denoted \trainingSet).

A probabilistic type is a set of counts over a set $\mathcal{T}$ of basic types.
Formally, we define a space of probabilistic types $\tilde{\mathcal{T}}$, where $\ptype \in \tilde{\mathcal{T}}$ has a form $\ptype=\{(\tau_1, c_1),\ldots,(\tau_n, c_n)\}$, such that $\tau_i \in \mathcal{T}$, $c_i \in \mathbb{Z}$. 
Every keyword $k \in \keys$ has a probabilistic type, expressed $k:\ptype$, as opposed to the basic type notation $k::\tau$.
The count for $(\tau_i,c_i) \in \ptype$ should be equal to the number of times a key in \trainingSet has a potential match to type $\tau_i$.
%We also denote the set of key-value pairs with $C$ and a keyword value pair that exists in some configuration file as $(k,v) \in C$.

Fig.~\ref{fig:ptypes} provides a calculus for type judgments for deriving equality rules with probabilistic types.
We use the notation $\ptype[\tau=N]$ to create a probabilistic type with the count of $N$ for $\tau$ in {\scriptsize PTYPE}.
This rule simply counts the regex matches for each key value pair.
In the {\scriptsize BOOL} judgment, a keyword with a probabilistic type $k : \ptype \in \tilde{\mathcal{T}}$ can be resolved to the basic type $bool$ when the $\ptype$ satisfies the predicate $p_{bool}$, \ie\ the probabilistic type has enough evidence.
The definition of enough evidence must be empirically determined by the user depending on the quality of the learning set.

{
\setlength{\abovecaptionskip}{-.05pt}
\setlength{\belowcaptionskip}{-15pt}
\begin{figure}
\begin{mathpar}
\hspace{-1cm} %no idea why, but this is better than \centering here
\inferrule* [Right=ptype]
{c_{int} = |\{\forall C \in \trainingSet.\ \forall (k,v) \in C.\ v \in \mathbb{Z}\}| \\ c_{bool} = |\{\forall C \in \trainingSet.\ \forall (k,v) \in C.\ v \in \{0,1,ON,OFF\}\}| }
{k : \ptype[int = c_{int}, bool = c_{bool}]}
\and\\
\inferrule* [Right=int]
{k:\ptype \\ p_{int}(\ptype)}
{k :: int}
\and
\inferrule* [Right=bool]
{k:\ptype \\ p_{bool}(\ptype)}
{k :: bool}
\and
\inferrule* [Right=eq\_rule]
{k_1 :: \tau \\ k_2 :: \tau}
{eq(k_1,k_2) :: Rule}
\end{mathpar}
\caption{Type judgments for a probabilistic type system with $\mathcal{T} = \{bool,int\}$ and an equality rule}
\label{fig:ptypes}
\end{figure}
}

In order to define predicates, we use the notation $|\tau_i\ptype|$ to select $c_i$ from a $\ptype \in \tilde{\mathcal{T}}$.
As an example, we can set the predicate $p_{bool}(\ptype)= |bool\ptype| \geq 3 \land |int\ptype| \leq 1$,
Then taking these type inference rules above, we can run an example inference on the sample training set in Figure~\ref{fig:tset}.
The resultant probabilistic type \texttt{bar}$:[bool=3,int=1]$ is then resolved to \texttt{bar}$::bool$.
 
A user may pick predicates for probabilistic type resolution that result in overlapping inference rules.
For example, if a user picks $p_{int}(\ptype)= |int\ptype| \geq 1$, then {\scriptsize INT} overlaps with {\scriptsize BOOL}.
To resolve the ambiguity in this case we must add a new rule {\scriptsize INTBOOL} that introduces a new type $intbool$.
This type is only a place holder to be used in the additional subtyping relations $intbool <: int$ and $intbool <: bool$.
These subtype relations allow the $intbool$ to take the place of either $int$ or $bool$ when determining if two keyswords may be compared in the {\scriptsize EQ\_RULE}.

\begin{mathpar}
\hspace{-0.8cm} %no idea why, but this is better than \centering here
\inferrule* [Right=intbool]
{k:\ptype \\ p_{bool}(\ptype) \\ p_{int}(\ptype)}
{k :: intbool}
\and
\inferrule* [Right=order\_rule]
{k1,k2 \in C \\ k1 \neq k2}
{ord(k1,k2) :: Rule}
\end{mathpar}

In the case that there is not enough evidence to resolve a probabilistic type to a basic type, no type-dependent rules may be learned over that keyword.
However, we are still able to learn rules such as {\scriptsize ORDER}, which do not require any resolved type. 




\section{Learner}
\label{sec-learn}

The goal of the learner is to derive rules from the intermediate representation of the training set generated by the translator.
We describe an interface to define the different classes of rules that should be learned.
Each instance of the interface corresponds to a different class of of configuration errors, such as missing entry errors, ordering errors, and integer correlation errors. 
These errors can cause total system failures, but can also be more insidious, for example slowing down the system only when the server load increases beyond a certain threshold.

\rahul{and we may even be able to say that these insidiuous issues are much
harder to pin down by the standard delta-debugging technique of starting a
system multiple times with different configuration settings}

A rule is an implication relationship, $X \implies p(X,Y)$, between two possibly empty sets of keywords $X,Y \subset \keys$, where $\keys$ is the set of unique keys from the training set and the predicate $p$ is the one of the classes of configuration errors (order, missing, etc).
Implicitly we interpret this to means that if the keywords $X\cup Y$ appear in a configuration file, we expect the predicate $p$ to hold.
The task of the learning algorithm is to transform a training set to a set of rules, weighted with \textit{support} and \textit{confidence}.
These two key metrics are taken from \textit{association rule learning}~\cite{agrawal1993mining}, a technique that can be summarized as inductive machine learning.
In the configuration verification domain, standard association rule learning is best suited to learn integer correlation and missing keyword rules.
In fact, a more specialized technique for learning ordering rules would be sequence mining~\cite{}, and again another approach may be a better fit for rules over a single keyword, as is needed for probabilistic types.
Since the full algorithm details are out-of-scope of this paper, here we only describe in detail the metrics which are most relevant to verification.

%TODO math def of support and confidence
Each power set of keywords, $\{X,Y\}$, is assigned a support and confidence measure during the learning process.
Support is the number of times the set of keywords in the proposed rule have been seen the in the training set.
Confidence is the number of times the rule predicate has held true over the given keywords.
In the learning process, these are given threshold, below which a rule will be reject for lack of evidence.

\para{Order}
Ordering errors take the form $X \implies order(X,Y)$, where $|X|,|Y|=1$ and the keyword $X$ must come before the keyword $Y$ in any configuration file.

\para{Missing}
$X \implies missing(X,Y)$, where $|X|,|Y|=1$ and the keyword $X$ must appear in the same file as the keyword $Y$ in any configuration file.

\para{Type}
The type rule is a set of rules over multiple predicates, which take the form $X \implies isType\ast(X)$, where $|X|=1, |Y|=0$ and $\ast$ matches all the basic types (string, int, etc).
In \app, probabilistic typing is implemented as an instance of the learning interface.
This module will specify the counting and resolution type judgments from Sec.~\ref{sec:ptypes}.
It 

\para{Integer Correlation}
\app supports two types of integer correlation rules, coarse-grained and fine-grained.
Coarse-grained rules follow $X \implies compare(X,Y)$, where $|X|,|Y|=1$ and $compare \in \{<,=,>\}$, such that $X$ must hold $compare$ to $Y$.
Fine-grained rules follow $X \implies compare(X,Y)$, where $|X|=2,|Y|=1$ and $compare \in \{<,=,>\}$, such that for $k_1,\ k_2 \in X,\ k1*k2$ must hold $compare$ to $Y$.
These rules also implement a typing judgment over the keywords probabilistic type.
To avoid learning too many false positives, we restrict this rule to either $size*int=size$, $int*size=size$, or $int*int = int$.
Without probabilistic typing, we would also learn, for example, $int*int=size$.
%TODO include exactly how many false positive we prune with types over our training set? Here or in eval?

%TODO Aaron - add to learinign module section describe implemetnation?

\app is primarily implemented in Haskell.
The source code for our implementation is available at {\em (URL omitted for blind review)}.
 
The learning and checking modules are developed to allow
for customized extensibility. In order to verify a configuration file
against a new type of error, a user only needs to provide a new type
that is an instance of the \textit{Attribute} typeclass (\ie, define some
functions over that type). 
In particular, for each type of error the users
wish to detect, they must implement three functions;
\lstinline{learn}, \lstinline{merge}, and \lstinline{check}.


\subsection{Checker}
\label{sec-checker}

With the learned rules generates by the learner module,
  \app checks whether any entry in a target configuration file violates the learned rules and constraints.
\app parses a verification target configuration file the same way employed in the translator for learning,
  obtaining a structured and typed representation.
Then, the checker applies the learner to build the set of relations observed in the file.
For any relation that violates a known rule, the checker will output the predicate and keyword sets associated with that rule, as well as the learned support and confidence values.
Since the tool is probabilistic, we provide the user with these values to determine if they rule must be satisfied in their application on a case-by-case basis.
For instance, the \texttt{key\_buffer} misconfiguration from Sec. \ref{ex:fine} will only be noticeable if the application experiences a heavy traffic load, so the user may choose to ignore this error if they are confident this will not be an issue.


\subsection{Implementation}



%
\section{Ranges}
\label{sec-learn-ranges}

We discuss the technique we use to detect anomalous values for numerical attributes. For a given test file, let $A$ be the set of attributes contained in the file. Let $A_n$ be the subset of attributes of $A$ which are numerical types. Then, for each attribute $a \in A_n$, we construct a vector $v_a$ of the values corresponding to attribute $a$ seen over the entire data-set. For each $v_a$, we compute an interval  $$[\hat{v_a} - 3*MAD(v_a), \hat{v_a} + 3*MAD(v_a)],$$ 
where $\hat{v_a}$ represents the median over the values in $v_a$ and $MAD(v_a$) refers to the median absolute deviation. This is a standard outlier detection test, namely the Hampel identifier. \footnote{Mathematically, $MAD(v_a) = 1.4826* median(|v_a - \hat{v_a}|)$, estimating standard deviation for a normal distribution.} If the value for a numerical attribute in a test file falls outside of this interval, a warning is printed to the user indicating the violating value, the attribute, and the upper or lower Hampel threshold. 



\section{Checker}
\label{sec-checker}

With the learned rules and constraints in hand (generated
by the learner module),
\app checks whether any entry in a target configuration file
violates the learned rules and constraints.
For a given configuration file, \app parses it using the same
way employed in the translator, thus obtaining a structured
and typed representation for the target configuration file.
Then, the checker uses two sub-modules to check the
following aspects of the target configuration file.

\para{Entry missing violation.}
\app 's checker can detect whether two or more entries should
appear together in the same configuration file based upon
entry missing rules generated by the learner.

\para{Ordering violation.}
\app can check whether two or more entries have the ordering 
problem, which means whether the order of some entries in the 
target configuration follows the rules output by the learner.

\para{Correlation violation.}
\app checks if the target configuration file follows the 
correlation rules (including fine-grained value correlation rules)
learned from the training set. The rule would be ignored 
if the involved entries are absent in the target configuration file.

\para{Data type violation.}
For each entry to be checked in the target configuration file,
the checker reads its type information inferred from the training set,
and uses the generated language model to verify whether the types
are matched. A type violation is reported if the verification
fails.

\para{Suspicious warning.}
This checking occurs in the second sub-module of the checker.
Different from the previous checking tasks,
suspicious warning is just to detect whether some value
is too different (or distinguished) from the same entries in the
training dataset. Even if some values are statistically different
from the ones in the training dataset, 
it does not mean such a value is incorrect;
thus \app, in this case, throw out a warning to the user who
enters the target configuration file, and a report containing 
normal values in the training dataset.
\app allows users to choose whether they want to change 
the values according to the ones in the training configuration
files or not.


\section{Discussion and Limitations}

\ennan{Of course, our tool has many limitations. For example, we cannot
detect the misconfiguration correlation across multiple individual
systems unwilling to share configuration information with each other.}

This section discusses a few \app's limitations
and possible solutions.

\para{Legal misconfigurations.}
While \app can check diverse configuration errors without
human participations, all the misconfiguration detection techniques,
including \app, cannot handle some types of configuration errors
resulting from events occurred during system runtime.
Such configuration errors are referred to as {\em legal misconfigurations}%
~\cite{yin11anempirical}. As shown in Yin {\em et al.}%
~\cite{yin11anempirical}, many parameter misconfigurations have 
perfectly legal parameters but do not deliver the functionality intended
by users. These cases are more difficult to detect by
automatic checkers and may require more user training or
better configuration design.
A potential solution is to combine existing misconfiguration diagnosis
tools, \eg, X-ray~\cite{}, with \app in order to enhance the 
misconfiguration checking capability.

\para{} 

\section{Implementation}

\app is primarily implemented in Haskell, and the suspicious values
detection module was written in R. 
%The suspicious value module reimplements the
%main functionality of the Haskell system, and is is a testament to the
%simplicity of the algorithm and the ease of reproduction. 
For clarity,
we only present the key components of the Haskell implementation here.

The translator is developed as a parser for the grammar
described in $\S$\ref{sec-trans}. 
%The implementation presented no interesting challenges. 

The learning and checking modules are developed to allow
for customized extensibility. In order to verify a configuration file
against a new type of error, a user only needs to provide a new type
that is an instance of the \textit{Attribute} typeclass (\ie, define some
functions over that type). 
In particular, for each type of error the users
wish to detect, they must implement three functions;
\lstinline{learn}, \lstinline{merge}, and \lstinline{check}.

\com{
\begin{lstlisting}
class Foldable t => RuleSet t a where
  learn :: IRConfigFile -> t a
  merge :: t a -> t a -> t a
  check :: t a -> IRConfigFile -> [Error]
\end{lstlisting}

This typeclass is polymorphic over the data structure as denoted by 
\lstinline{Foldable t =>}, 
which means user can choose any data structure they prefer.
In general, the best data structure is a hashmap 
from key-value pairs to relations, 
previously defined as a \textit{P\_Rule} in $\S$\ref{subsec-rules}.
}

The \lstinline{learn} function takes the intermediate representation of 
a single configuration file (as generated by the translator), 
and generates the set of rules that can be derived from that file.
Specifically, these rules take the form of 
a \textit{P\_Rule} to track the counts of relation events.
The \lstinline{merge} function will unify two sets of rules, from two separate configuration files, into a single consistent set of rules. 
For the probabilistic rules, this tends to be simply a union between the two sets. 
When two equivalent rules are merged, the relation event counts should be summed to update the probability distributions.

The \lstinline{check} function will start all the rules we have learned, 
and filter until we only have the rules that have been broken by the file 
of interest. We first eliminate the rules 
that do not pass the probability check defined in $\S$\ref{subsec-rules}.
Second, we only consider the rules which are relevant to the file -- we 
do not need to check rules about the entries 
that do not appear in the file.
Lastly, we take the errors 
as any rule that does not hold over the given file.


\com{
A sketch of a typical implementation of these three functions is shown below. Again, the specifics of this implementation will vary based on the type of error being detected. 

\begin{lstlisting}
learn f = 
  count_relation_events (all_line_pairs f)
merge s1 s2 = 
  unionWith sum s1 s2
check rs f = 
  likely_rs  = filter prob_check rs
  related_rs = filter applies_to_file likely_rs
  errors_rs  = filter (is_valid f) related_rs
 in errors
\end{lstlisting}
}

%The rest of the system is structural code to handle these modules for the user. 
The core learning algorithm will learn over each file individually, 
then merge the results of learning together. 
Since we learn each set of rules on each file in isolation from the others, we have a pleasingly parallel situation.
The \lstinline{learn} function can then be called in parallel over the entire learning set.
As a functional language, there is no programming overhead to implement
this in Haskell -- we simply import the parallel mapping library \cite{parallel} to replace all uses of  \lstinline{map} with \lstinline{parmap}.
The merge stage could also be parallelized by using a divide and conquer approach, but this is not a priority since the learning stage only needs to be run once per learning set, and can then be cached (in our case, as a .json file) for fast reading when doing verification.

The source code for our implementation is available at {\em (URL omitted for blind review)}.


\section{Evaluations}
\label{sec-eval}

We conduct three experiments to evaluate our \app
prototype. Our experiments aim to answer the following
questions.

\begin{itemize}

\item Whether \app can successfully 
  detect real-world configuration errors?

\item Whether \app can correctly detect anomalous values?

\item How long do training and verification last?

\end{itemize}


\begin{table*}[tbp]
\centering
\caption{Sampled incorrect configuration files for 
misconfiguration detection evaluation}
\label{table-casestudy}
\begin{footnotesize}
\begin{tabular}{|l|l|l|l|}
\hline
{\bf ID} & {\bf Problem Description} & {\bf Error Type} & 
{\bf \app Report}  \\ 
\hline
\hline
1 & MySQL cannot create test file  
& Entry Missing 
& Expected innodb\_force\_recovery=1 in the \\ & & 
&  same file as: innodb\_strict\_mode=1\\ \hline

2 & How can my wait\_timeout/interactive\_timeout  
& Entry Missing
& Expected set\_time=1 in the same\\ & be ignored 
& & file as wait\_timeout \\ \hline

3 & MySQL on EC2 instance becomes very slow
& Correlation
& Expected key\_buffer\_size $>=$ \\ & after the first query
&& thread\_size * sort\_buffer\_size \\ \hline

4 & MySQL cannot successfully make  
& Type Error
& Expected master-host[mysqld] should 
 \\ & replication &&  be a valid IP address\\ \hline

5 & MySQL running on production servers has
& Type Error
& Expected set-variable[myisamchk] \\ &  gone away error in memory
&&  should be a memory size like 20M \\ \hline

6 & Cannot start MySQL 5.5 as normal 
& Ordering
& Expected socket[mysqld]=/var/lib/ \\ & user Fedora 15 
&&  mysql/mysql.sock BEFORE  \\ & && user[mysqld]=mysql

\\ \hline

7 & Fail to login MySQL
& Type Error
& Expected old\_passwords[mysqld] 
 \\ &&&  should be a valid value \\ \hline

8 & MySQL running at CentOS cannot know 
& Ordering
& Expected localhost before bind-
 \\ & enable named pipe
&&  address[mysqld] \\ \hline

9 & Troubles installing MySQL5 via  Darwin 
& Ordering
& Expected port[mysqld]=3306 BEFORE \\ & Ports
&&  socket[mysqld]=/tmp/mysql.sock \\ \hline

10 & Fail to install Percona xtraDB 
& Entry Missing
& Expected wsrep\_node\_address[mysqld] \\ & cluster on ubuntu 13.04
&&  in the same file as: wsrep\_sst\_method \\ &&& [mysqld] \\ \hline

11 & MySQL access denied for user rootlocalhost 
& Ordering
& Expected port[mysqld]=3306  \\ & &&   BEFORE socket[mysqld]= \\ &&& /tmp/mysql.sock
\\ \hline

12 & MySQL Partition Problem
& Correlation
& Expected innodb\_flush\_max  \\ & &&  \_commit[mysqld] $>=$ innodb \\ &&&  \_support[mysqld] \\ \hline

13 & mysql\_upgrade script problems on 
& Type Error
& Expected log-slow-queries[mysqld]  \\ & MySQL 5.0.24 
&&  should be a path type\\ \hline


14 & MySQL: bug report!!
& Ordering
& Expected innodb\_file\_io\_threads[mysqld]  \\ & 
&& BEFORE innodb\_log\_files\_in\_group[mysqld]\\ \hline

15 & MySQL max connections changed 
& Correlation
& Expected innodb\_file\_per\_table \\ & without notice
&&  [mysqld] $<=$ max\_connections[mysqld]  \\ \hline

16 & Enabling log in MySQL 5 6 
& Type Error
& Expected sql\_mode[mysqld] should  \\ & prevents server from starting
&&  be a string \\ \hline

17 & MySQL needs help for rapidly growing 
& Correlation
& Expected innodb\_buffer\_pool\_size \\ & table and decreasing speed
&&  [mysqld] $>$ sort\_buffer\_size[mysqld] \\ \hline

18 & MySQL filled all RAM on the system
& Correlation
& Expected join\_buffer\_szie * \\ & && max\_connections $<=$ key\_buffer\_size \\ \hline


19 & MySQL hit a max limit
& Correlation
& Expected join\_buffer\_size * \\ &&& max\_connections $<=$ key\_buffer\_size \\ \hline

20 & Cannot create a new thread
& Correlation
& Expected join\_buffer\_size * \\ &&& max\_connections $<=$ key\_buffer\_size \\ \hline

\end{tabular}
\end{footnotesize}
\end{table*}




\subsection{\app's Effectiveness}

We now apply \app to check against real-world misconfiguration problems.
We use a real misconfiguration dataset~\cite{xu15hey}
that are collected from several online forums, \eg, 
Stack Overflow and MySQL forum.
Each configuration file in this dataset was posted by 
programmers or system administrators, 
when they confront the misconfiguration issues in practice.
This dataset contains 261 incorrect MySQL configuration files.

We apply \app to learn and check the configuration files in this
dataset. Then, we observe how many errors \app can detect,
and manually check whether these detected errors are indeed
configuration errors.
For all the 261 configuration files, we found \app is
able to report all the errors correctly.
For 217 of the 261 configuration files, we found \app's 
error detection results are correct and accurate,
\ie, correctly report the real misconfiguration problem.
However, for 44 of them, \app outputs relatively high false positives,
\eg, higher than 50 false positives,
which is not very helpful to users in practice.
The reason is our training set contains too many incorrect configuration
files, which produce a lot of noises to our learning process,
thus resulting in so many false positives.
  
In order to clearly evaluate the effectiveness of \app,
we extract 20 incorrect configuration files from the dataset.
As a case study, we present the problem descriptions of these files, 
the types of their errors, and the outputs produced by \app.
Table~\ref{table-casestudy} details this information.

As shown in Table~\ref{table-casestudy},
we found \app is capable of detecting many configuration errors previous
efforts, \eg, EnCore~\cite{zhang14encore}, cannot deal with.
To the best of our knowledge, no existing effort
is able to detect fine-grained value correlation errors in the 
MySQL configuration files.
For example, the third case in Table~\ref{table-casestudy}
presents \app detects the fine-grained correlation,
{\tt key\_buffer\_size} should be larger than 
{\tt sort\_buffer\_size} * {\tt max\_connections}.


\begin{table*}[t]
\centering
\caption{Sampled benchmarks for anomaly detection}
\label{table-anomaly}
\begin{small}
\begin{tabular}{|l|l|l|l|}
\hline
{\bf ID} & {\bf Problem Description} & {\bf URL} & 
{\bf \app Report}  \\ 
\hline
\hline
1 & MySQL Server has gone away error in Wamp 
& goo.gl/axnezi  
& WARNING: Violated Upper Hampel Rule for 
\\ &  & 
& {\tt max\_allowed\_packet} with value 104857600 
 \\ \hline

2 &  MySQL has abnormally high load for CPU 
& goo.gl/JrRLrR
& WARNING: Violated Upper Hampel Rule for  
\\ & & 
& {\tt sort\_buffer\_size} with value 1048576000 \\
& & & WARNING: Violated Upper Hampel Rule for  
\\ & & 
& {\tt read\_rnd\_buffer\_size} with value 283115520 \\ \hline

3 & User is having performance issues with MySQL 
& goo.gl/34jTB5
& WARNING: Violated Upper Hampel Rule for  
\\ & & 
& {\tt query\_cache\_limit} with value 134217728 \\ \hline

\end{tabular}
\end{small}
\end{table*}

\subsection{Detecting Anomalous Values}

The purpose of checking anomalous values in configuration files 
is to avoid potential performance or workload problems. 
Checking anomalous values is mainly the responsibility of
the second sub-module checker of \app (see $\S$\ref{sec-checker}).

We run \app to check 30 real-world benchmarks from Stack Overflow 
website, and found \app is able to report anomalous values in
these configurations. We also manually check whether these
anomalous values are correct by comparing our results with
the answers on Stack Overflow.

Table~\ref{table-anomaly} shows three examples.
We present the problem description, the link and the output results
of \app. We found these anomalous values are very hard for
administrators or users to detect, because they look correct.
However, as shown in the links posted in Table~\ref{table-anomaly},
such anomalous values typically lead to critical performance
problems in practice.

\subsection{The Run-Time of Verification}

\ennan{Here, we need three pictures. In the first figure, x-axis should
be the number (or the size) of entries in the training dataset, and 
y-axis should be the run-time of parsing. In the second figure,
we need to measure the time of generating rules. }


\section{Related Work}

Providing language support has been considered as a promising means
of tackling configuration problems~\cite{xu15systems}.
Nevertheless, practical language-based misconfiguration
detection approach still remains an open problem.

\para{Configuration languages.}
There have been several language-support efforts proposed to prevent
configuration errors introduced by fundamental deficiencies in
either untyped or low-level languages. For example, in network
configuration management area, it is easy for administrators to
produce configuration errors in their routing configuration files.
PRESTO~\cite{enck07configuration} 
automates the generation of device-native configurations
with configlets in a template language. 
Loo {\em et al.}~\cite{loo05declarative} adopt Datalog to reason about 
routing protocols in a declarative fashion. 
COOLAID~\cite{chen10declarative} constructs
a language to describe domain knowledge about network devices and
services to convenient network reasoning and management.

Compared with these existing efforts, 
our work mainly focused on software systems, \eg, MySQL and Apache,
rather than network configurations. In addition, we do not need 
the user of \app to manually write a configuration file with the proposed
language, since \app can automatically parse a target configuration
file into our proposed representation.

\para{Misconfiguration detection.}
Misconfiguration detection techniques aim at checking the configuration
efforts before the system outages occur.
Most of existing detection approaches check 
the configuration files against a set of predefined correctness 
rules, named constraints, and then report the errors if 
the checked configuration files do not satisfy these rules.

Huang {\em et al.}~\cite{huang15confvalley} proposed a specification 
language, ConfValley, to validate 
whether given configuration files meet administrators' 
``belief'' in mind. Different from \app, ConfValley itself does not
have inherent misconfiguration checking capability, since it only offers
a language representation. In addition, administrators have to
manually write specifications with ConfValley, which is an error-prone
process; on the contrary, \app does not need users to manually
write anything.

Several machine learning-based misconfiguration detection efforts 
also have been proposed~\cite{yuan11context, zhang14encore}.
EnCore~\cite{zhang14encore} is the most close work to \app.
It introduces a template-based
learning approach to improve the accuracy of their learning results.
The learning process is guided by a set of predefined rule templates
that enforce learning to focus on patterns of interests.
By this way, EnCore filters out irrelevant information and reduces
the false positives; moreover, the templates are able to express
system environment information that other machine learning
techniques cannot handle.
Compared with EnCore, \app has the following advantages.
First, \app does not rely on whether the files in given configuration set 
are 100\% correct. Second, \app not only can cover much more types of 
misconfigurations, but also introduces probabilistic type.
Finally, \app is a language framework, which could 
even be used to write configuration files, but EnCore is only a 
misconfiguration detection tool.

\para{Misconfiguration diagnosis.}
Many misconfiguration diagnosis approaches have been proposed%
~\cite{attariyan10automating, attariyan12x-ray}.
For example, ConfAid~\cite{attariyan10automating} 
and X-ray~\cite{attariyan12x-ray} use dynamic information
flow tracking to find possible configuration errors that may result in
failures or performance problems. AutoBash~\cite{su07autobash} 
speculatively
executes processes and tracks causality to automatically fix 
misconfigurations. Different from \app, most of misconfiguration
diagnosis efforts aim at finding out errors after the system
failures occur, which typically lead to prolonged recover time.

\para{Misconfiguration tolerance.}
There have been several efforts proposed to test whether systems are 
tolerant to misconfigurations~\cite{xu13do}. 
%ConfErr~\cite{} uses a human error hodel from psychology and
%linguistics to inject misconfigurations into systems.
SPEX~\cite{xu13do} takes a white-box testing approach to automatically
extract configuration parameter constraints from source code and generates 
misconfigurations to test whether systems can tolerate the potential
configuration errors.

Making systems gracefully handle misconfigurations and eliminating
configuration errors are two orthogonal directions.
The former helps improve the robustness of systems and make 
diagnosis easier. This is especially important for 
software that will be widely distributed to end users.
Our work belongs to the latter case, which is used to 
prevent configuration errors before the system failures occur.


\section{Conclusion}

In this paper, we introduce \app, a highly modular framework 
that allows automatic verification of configuration files.
\app employs a translator to parse a sample configuration dataset
into a well-structured and typed intermediate representation,
and then uses a learner module to derive rules, 
thus building a language model.
For a given configuration file we want to verify,
\app uses the generated language model to check
whether any rule is violated.
We evaluate \app using a real-world dataset~\cite{configdataset}
which contains 261 incorrect MySQL configuration files.
Our experimental result shows \app is able to
correctly detect errors in 217 files in that dataset.

Our \app prototype still has many limitations. 
For example, we cannot handle configuration errors that can be 
triggered during system execution time, 
and we cannot detect misconfiguration across software components
maintained by different parties. 
Nevertheless, we believe \app represent a practical path
toward automatic language-based configuration verification.



%\newpage


\bibliographystyle{plain}
\bibliography{os}


\end{document}


