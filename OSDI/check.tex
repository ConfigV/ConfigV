
\section{Checker}
\label{sec-checker}

With the learned rules and constraints in hand,
\app checks whether any entry in a target configuration file
violates the learned rules.
For a given configuration file, \app parses it using the same
way employed in the translator, thus obtaining a structured
and typed representation for the target configuration file.
Then, the checker uses two sub-modules to check the
following aspects of the target configuration file.

\para{Ordering violation.}
\app can check whether two or more entries have the ordering 
problem, which means whether the order of some entries in the 
target configuration follows the rules output by the learner.

\para{Correlation violation.}
\app checks if the target configuration file follows the 
correlation rules (including fine-grained value correlation rules)
learned from the training set. The rule would be ignored 
if the involved entries are absent in the target configuration file.

\para{Data type violation.}
For each entry to be checked in the target configuration file,
the checker reads its type information inferred from the training set,
and uses the generated language model to verify whether the types
are matched. A type violation is reported if the verification
fails.

\para{Suspicious warning.}
This checking occurs in the second sub-module of the checker.
Different from the previous checking tasks,
suspicious warning is just to detect whether some value
is too different (or distinguished) from the same entries in the
training dataset. Even if some values are statistically different
from the ones in the training dataset, 
it does not mean such a value is incorrect;
thus \app, in this case, throw out a warning to the user who
enters the target configuration file, and a report containing 
normal values in the training dataset.
\app allows users to choose whether they want to change 
the values according to the ones in the training configuration
files or not.
