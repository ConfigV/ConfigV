\section{Motivating Examples}
\label{sec-motiv}

In this section we present the capability of \app through 
detecting errors in several real-world misconfiguration examples. 
These examples were non-trivial configuration errors
that were reported on StackOverflow~\cite{stackoverflow},
a popular question and answer website for programmers and administrators. 
%To better understand these problems, 
%we explored and analyzed misconfigurations 
%on a large number of user forums and on-line discussion sites.

\para{Example~1: Ordering error.} 
Ordering errors were reported by Yin {\em et al.}~\cite{yin11anempirical} and our first
example illustrates how ordering errors can cause a system to crash. When a user configures PHP 
to run with the
Apache HTTP Server, most likely the user will take some already existing configuration files and adapt them
to suit her needs. The configuration file might contain, among others, 
the following lines:

\begin{lstlisting}[language=C, xleftmargin=.01\textwidth]
    extension = mysql.so
        ...
    extension = recode.so
\end{lstlisting}

This configuration file will cause the Apache server to 
fail to start due to a segmentation fault error. 
This is because, when using PHP in Apache, the extension {\tt mysql.so} 
depends on {\tt recode.so}, and their relative ordering
is crucial. 
We call the above example of a misconfiguration file
an {\em ordering error}.
Yin {\em et al.} report that ordering errors widely exist in
many system configurations, \eg, PHP and MySQL,
and typically lead to multiple system crash events.
However, no existing tool can effectively solve 
or detect this problem~\cite{zhang14encore, xu15systems, xu13do}.

By invoking \app, the user can detect such a configuration error.
In particular, \app reports that {\tt recode.so} 
should appear before {\tt mysql.so}, as shown
below: \ennan{Mark, please update all the motivating examples'
output results according
to our newest version implementation.}

\begin{lstlisting}[language=C, xleftmargin=.01\textwidth]
    ORDERING ERROR: Expected "extension" "recode.so"
    BEFORE "extension" "mysql.so"
\end{lstlisting} 

\para{Example~2: Fine-grained value correlation error.} 
Our second misconfiguration example~\cite{correlation} 
comes from a discussion on StackOverflow.
The user has configured her MySQL as in the following:

\begin{lstlisting}[language=C, xleftmargin=.01\textwidth]
    key_buffer_size = 384M
    max_heap_table_size = 128M
    max_connections = 64
    thread_cache_size = 8
        ...
    sort_buffer_size = 32M
    join_buffer_size = 32M
    read_buffer_size = 32M
    read_rnd_buffer_size = 8M
        ...
\end{lstlisting} 

The user complains that her MySQL load was very high, 
causing the website's
response speed to be very slow.
In this case, {\tt key\_buffer\_size} is used by all the threads
cooperatively, while {\tt join\_buffer} and {\tt sort\_buffer} are 
created by each thread for private use; thus, the maximum amount
of used key buffer, \ie, {\tt key\_buffer\_size}, should be larger than 
{\tt join|sort\_buffer\_size} * {\tt max\_connections}. 
Clearly, in the above example, it does not hold, 
so this misconfiguration causes MySQL to load very slowly.

If we run \app on this configuration file, \app  would return:

\begin{lstlisting}[language=C, xleftmargin=.01\textwidth]
    INTEGER RELATION ERROR:
    Expected "key_buffer_size" >= "max_connections" * "sort_buffer_size"
\end{lstlisting} 

We can see that during the learning process not only do we learn simple statements that compare two values, 
but also we learn more complex correlations. We call these more complex relations fine-grained value correlations, and the errors
\emph{fine-grained value correlation errors}. 
This type of error is more sophisticated than the simple value correlation that some tools can detect~\cite{yin11anempirical, zhang14encore}.
A typical value correlation error
states that one entry's value should have a certain correlation with
another entry's value. For example, in MySQL,
the value of {\tt max\_connections} should be higher than
{\tt mysql.max\_persistent}. 

Our tool can learn this simple correlation as well, but learning more complex properties requires a different 
approach to the learning 
process. We need to track several variables. It is not enough just to compare them, but we need to learn, as in this particular case, which
invariant must be preserved. It was pointed out by Xu {\em et al.}~\cite{xu15hey} that
detecting fine-grained value correlation errors 
is a much more challenging
task than the normal value correlation problem.
To the best of our knowledge, \app is the first tool that
is able to check such fine-grained value correlation problems.

\para{Example~3: Missing entry error.} 
Many critical system outages result from the fact that an important
entry was missing from the configuration file. 
We call such a problem a {\em missing entry error}.
In a public misconfiguration dataset~\cite{configdataset},
many MySQL failure reports were caused by
missing entry errors.
Below is a real-world missing entry error example~\cite{yin11anempirical}:
when a user wants to use OpenLDAP to enable her directory access
protocol, she needs to use the password policy overlay. This is usually
achieved via the following entries in the OpenLDAP configuration file:

\begin{lstlisting}[language=C, xleftmargin=.01\textwidth]
    include schema/ppolicy.schema
    overlay ppolicy
\end{lstlisting} 

When using the password policy overlay in OpenLDAP, 
users must first include the related schema.
Leaving out the {\tt include schema/ppolicy.schema} entry, 
as done by many users~\cite{yin11anempirical}, 
causes the failure of LDAP. 
If the user runs \app on such a misconfiguration file,
\app would return:

\begin{lstlisting}[language=C, xleftmargin=.01\textwidth]
    MISSING KEYWORD ERROR: Expected "overlay" "ppolicy"
    in the same file: "include" "schema/ppolicy.schema"
\end{lstlisting} 

\para{Example~4: Type errors.} 
Many system performance problems are caused by 
assigning incorrect type of values to some key in configuration
files. Consider the following real-world misconfiguration file: 
a user tries to install MySQL and she needs to initiate the path
of the log information generated by MySQL.
This user puts the following assignment in her MySQL
configuration file: 

\begin{lstlisting}[language=C, xleftmargin=.01\textwidth]
    general_log = /var/log/mysql/mysql.log
\end{lstlisting} 

Unbeknowest to this user, the entry ``general\_log'' should be an 
integer, not a string. In MySQL, there is another entry named
``general\_log\_file'' used to specify the log path.
With \app, this user can get the following result:

\begin{lstlisting}[language=C, xleftmargin=.01\textwidth]
    TYPE ERROR: Expected a Int with P=1.0 for
    "general_log [mysql]"
\end{lstlisting} 

%The above result means that we need to assign an integer value to
%the entry ``general\_log''. 
