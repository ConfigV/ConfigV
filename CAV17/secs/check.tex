
\section{Checker}
\label{sec-checker}

With the learned rules and constraints in hand (generated
by the learner module),
\app checks whether any entry in a target configuration file
violates the learned rules and constraints.
For a given configuration file, \app parses it the same
way employed in the translator, thus obtaining a structured
and typed representation for the target configuration file.
Then, the checker uses two sub-modules (shown in 
Figure~\ref{fig-overview}) to check the target
configuration file based on rules and constraints.

\para{Error detecting.}
The first sub-module of our checker,
named error detecting in Figure~\ref{fig-overview}, 
is able to detect the following errors:
missing entry errors, ordering errors, 
correlation errors (including fine-grained value correlation errors),
type errors, and system environment errors (depending on templates).
In particular, the checker simply sees whether the representations 
parsed from the target configuration files violate our
learned rules.

\para{Anomaly checking.}
This checking occurs in the second sub-module of the checker.
Different from the previous checking tasks,
a suspicious warning detects whether some value
is overly different from the same entries in the
training dataset. Even if some values are statistically different
from the ones in the training dataset, 
it does not mean such a value is incorrect;
thus \app, in this case, throw out a warning to the user who
enters the target configuration file, and a report containing 
normal values in the training dataset.
\app allows users to choose whether they want to change 
the values according to ones in the training configuration
files.
