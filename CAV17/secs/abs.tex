
\begin{abstract}

We verify configuration files.
While previous attempts to learn a model for these highly unstructured files have focused on machine learning, we instead take a probabilistic logic approach.
Probabilistic logic allows us to provide justification for our learning results, giving proof of the verification task.
Because the resulting model has a clear logical structure, as opposed traditional machine learning techniques such as neural nets, we can do further analysis on the model that improves our results.

\app is capable of detecting various errors that cannot
be detected by previous efforts,
including entry ordering errors, fine-grained value correlation errors, 
and missing entry errors. 

System failures resulting from configuration errors 
are major reasons for compromised availability and
reliability of today's software systems.
Although many misconfiguration handling techniques
such as checking, troubleshooting, and repair
have been proposed, 
offering automatic verification for configuration files -- as often  
done for regular programs -- is still an open problem.
This is because software configurations are typically written in
poorly structured and untyped ``languages'', and 
specifying constraints and rules for configuration 
verification is non-trivial in practice.

This paper presents \app, the first automatic verification framework for
general software configurations.
\app verifies a target configuration file $F$ through three steps.
Firstly, \app analyzes a dataset containing many sample configuration 
files belonging to the same system as $F$,
translating these sample files to a
well-structured and probabilistically-typed 
intermediate representation.
Secondly, \app derives rules and constraints by analyzing
this intermediate representation, thus building a
sophisticated language model.
Finally, \app uses the resulting language model to verify $F$.
The \app framework is highly modular, 
does not rely on system source code, and
can be applied to any new configuration file type with minimal user input. 

\end{abstract}
