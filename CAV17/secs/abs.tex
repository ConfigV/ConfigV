
\begin{abstract}

System failures resulting from configuration errors 
are one of the major reasons for compromised reliability of today's software
systems. Although many techniques have been proposed for a 
configuration error
detection, these approaches mainly can be applied after an error has occurred. Verifying configuration files is, nevertheless, 
a challenging problem,
because 1) software configurations are typically written in
poorly structured and untyped ``languages'', and 
2) specifying rules for configuration 
verification is challenging in practice.
This paper presents \app, the first automatic verification framework for
general software configurations.
Our framework works as follows: in the pre-processing stage, we first automatically derive a specification.
Once we have a specification, we check if a given configuration file adheres to that specification.
The process of learning specification works through three steps.
First, \app parses a training set of configuration 
files (not necessarily all correct) into a
well-structured and probabilistically-typed 
intermediate representation.
Second, based on the association rule
learning algorithm \app learns rules from 
these intermediate representations. These rules are 
establishing relationships between the keywords appearing in the files. 
Finally, \app employs rule graph analysis to refine the 
resulting rules. \app is capable of detecting various configuration errors,
including ordering errors, integer correlation errors, type errors,
and missing entry errors. We evaluated \app by verifying 
public configuration files on Github, and we show that \app can 
detect known configuration errors in these files.

\end{abstract}
