\section{Learner}
\label{sec-learn}


The goal of the learner module is to derive rules and constraints from
the intermediate representation generated by the translator.
In general, the learner module has two components.
The first component ($\S$\ref{subsec-rules}) 
learns rules for checking configuration errors like
missing entry errors, ordering errors, and fine-grained value correlation errors. 
These errors tend to cause total system failures.
%Once the configuration file has been validated against such rules, 
%the user may choose to invoke a more sensitive constraint checker. 
The second component ($\S$\ref{subsec-constraints}) 
aims to derive 
constraints on entries to check for suspicious (or anomalous) values 
that may violate standard practice. These anomalies can cause partial 
degradation of the system, 
such as significant reduction in performance, or even 
total failure as in Example~4 of $\S$\ref{sec-motiv}.

\subsection{Derivation of Probabilistic Rules}
\label{subsec-rules}

The first component learns two types of rules: rules that are inferred using templates associated with types 
and rules which we call \emph{untyped specifications}. An example of an
untyped specification is the ordering or missing entry constraint.
%Both are rules that must hold over multiple parts of a configuration file. 

\para{A strawman solution.}
We first present a strawman solution (employed by
previous work~\cite{santolucitoCAV}) that uses 
a set of correct configuration files as a learning set, 
from which it is possible to derive rules 
that must hold with absolute certainty. 
In practice, however, it is difficult to obtain a set of files 
that is both guaranteed to be without errors
and large enough to learn sufficiently many rules.
This usually requires manual verification of the learning set, 
which is prone to error.

As a result of this restriction, 
these efforts only consider a rule if it holds over exactly every file in 
the learning set. This behavior can be formally described as follows:
\begin{small}
\begin{align*}
C :=&\ \text{Correct Learning Set}\\
LR :=&\ \text{Learned Rules} :: \{\textrm{Rule}\}\\
RR :=&\ \text{Reported Rules} :: \{\textrm{Rule}\}\\ 
C =&\ \text{\{Configuration Files in Intermediate Representation\}}\\
LR =&\ \{ r\ \mid \forall file \in C,\ holds(r,file)\} \\
RR =&\ \{ r\ \mid r \in LR \ \land \neg\ holds(r,\textrm{user file}) \}\\
\end{align*}
\end{small}
In the above, the ":: \{Rule\}" notation indicates that $LR$, the learned rule set, 
and $RR$, the reported rule set, are sets of rules.
A rule is only committed to the $LR$ learned rule set if 
the rule holds on all files in the training set $C$, as denoted by the $\forall$ quantifier. 
A broken rule is reported to the user in the $RR$ rule set if the rule is in $LR$, 
but does not hold on the input user file.

In general, each rule can be viewed as a mapping from a tuple of 
entry keys $(k_1, \ldots, k_n)$ in the intermediate representation of a 
configuration file to a probability function, $P_{rel}$. This can be represented by a rule
$r := (k_1, \ldots, k_n) \rightarrow P_{rel}$.
$P_{rel}$ is a probability function specific to the relation to assess.
The problem with the strawman solution is that the probability functions are restricted to be 
boolean, which is to say that the relation can either be unsatisfied or satisfied, either 0 or 1. 
For instance, for the rule that {\tt max\_connections} must 
be greater than {\tt mysql.max\_persistent}, the rule is denoted by
({\tt max\_connections}, {\tt mysql.max\_persistent}) $\to 1$, where 
$B_{>}$ is set to 1. In our approach, we permit the probability functions to take on any value from 0 to 1.

\para{Probabilistic approach.}
In \app's learner module, the rule learning mechanism is tolerant 
enough to accept a dataset of possibly incorrect configuration files. 
Rather than manually correcting each file, 
we extend the previous formalism to run probabilistic learning
on our intermediate representations (generated by the translator). 

Our probabilistic approaches for learning missing entry, 
ordering, and fine-grained value correlation rules stem 
from existing work with building 
non-probabilistic versions of these rule-learning algorithms. 
For each of these rules, 
we consider all possible permutations of keys that appear in every 
file which are appropriately typed, and for our learning process, calculate the likelihood that each of 
these permutations constitute a rule. 
Rule patterns in the example set that appear frequently are accepted,
so they can be used to evaluate new files.

More formally, this approach can be defined as follows:

\begin{small}
\begin{flalign*}
I\ \ :=&\ \text{Incorrect Learning Set}\\
LP :=&\ \text{Learned Probabilistic Rules :: \{P\_Rule\}}\\
RP :=&\ \text{Reported Probabilistic Rules :: \{P\_Rule\}}\\
I =&\  \text{\{Configuration Files in Intermediate Representation\}}\\
(\Pi, LP) =&\ \text{learnRules}(I)\\
RP =&\ \{ r\ \mid r \in LP\ \land \Pi(r)>p \land \neg holds(r, userfile) \}
\end{flalign*}
\end{small}

Note that the formalism relies on Algorithm~\ref{alg:plearn}. The algorithm considers both typed templates and
untyped specifications. As a similar iteration logic is used for both, an $opt$ flag distinguishes between the two situations.
constructRule in Algorithm~\ref{alg:crules} refers to the creation of a rule from a tuple of entries, 
$m = (e_1, e_2, \ldots, e_i, \ldots, e_{n-1})$.
The rule creation is formally represented as $m(k(e_1), k(e_2), \ldots, k(e_i), \ldots, k(e_{n-1}))$. 
In order to determine if the rule satisfied, the relation associated with $m(v(e_1), v(e_2), \ldots, v(e_i), \ldots, v(e_{n-1}))$ is evaluated for truth.
The rule can be template-generated, for instance {\tt max\_connections} > {\tt mysql.max\_persistent}, or
it can be untyped-specification-generated, for instance the ordering that {\tt recode.so} must come before 
{\tt mysql.so}.

The candidate entry set $Q$ in Algorithm~\ref{alg:plearn} 
differs for the two cases.
For typed templates, due to typing restrictions, 
we must first filter and examine entries associated with the same type, in order to check that 
the template is satisfied over appropriately typed argument entries. This manifests itself via $Q = filter(F,\tau)$
in Algorithm~\ref{alg:plearn}. For
untyped specifications, we do not need to adhere to this typing restriction, so we simply set $Q = F$.

%probability set associated with the rules, $\Pi$, contains values
%ranging from 0 to 1, which are then compared against an acceptance threshold for the final output.
%\[
%\{ P\_Rule = (a_j, a_k) | j \neq k \} \rightarrow \{ (R_1, R_2, ... , R_n) \}
%\]

%We can think of each of these $(a_j, a_k)$ as possibly having a different relationship, defined by the set $\{ R_i \}$, which cover the entire outcome space of possible relationships between the two values $(a_j, a_k)$.

%For the entry missing rules, we define $R_1$ as the event that $a_j$ and
%$a_k$ appear together, and $R_2$ to be the event that $a_j$ appears
%without $a_k$, or by the transitive equivalent, $a_k$ appears without
%$a_j$. For the ordering rules, we define $R_1$ as the event that
%$a_j$ appears before $a_k$ and $R_2$ be the event that $a_k$ appears
%before $a_j$. For the value correlation rules, we define $R_1$ as the
%event that $a_j \leq a_k$, $R_2$ the event that $a_j = a_k$, and $R_3$
%the case that $a_j \geq a_k$. Notice that the $R_i$ do not have to be
%disjoint, but only have to union to the entire probability space.

%By examining the learning set, we will derive a distribution of the set $\{R_i\}$ based on how many times we observe an occurrence of each relation. This distribution will then be used at checking time to determine if a user's configuration has broken a likely rule. 

A rule $r$ will be reported as broken if the probability the rule is
correct, $\Pi(r)$, is greater than some user defined constant, $p$. This
constant can be adjust to the user's preference. A small $p$ will
increase the likelihood of finding an error, but also increase the
number of false positives that are reported.

\subsection{Learning Suspicious Constraints}
\label{subsec-constraints}

With a configuration file that has been verified against catastrophic
failures (\eg, missing entry, type, and ordering errors), 
the user may also want to examine more subtle issues.
Anomalous values can cause tricky, but impactful, performance and memory
issues that are hard to debug, as discussed in Example 4 of 
$\S$\ref{sec-motiv}. 
Consequently, anomalous values should be flagged and a warning returned
to the user indicating the violation.

We now describe the technique we use to detect anomalous values for 
numerical attributes. Let $A$ be the set of attributes contained in the 
configuration files in the sample dataset. 
Let $A_n$ be the subset of attributes of $A$ which are numerically typed. 
Then, for each attribute $a \in A_n$, we construct a vector $v_a$ of the 
values corresponding to attribute $a$, seen over the entire sample dataset.
For each $v_a$, we compute 
an interval  $$[\hat{v_a} - 50*MAD(v_a), \hat{v_a} + 50*MAD(v_a)],$$ 
where $\hat{v_a}$ represents the median over the values 
in $v_a$ and $MAD(v_a$) refers to the 
median absolute deviation. 
This is a variant of a standard outlier detection test, namely the Hampel identifier.\footnote{Mathematically, $MAD(v_a) = 1.4826* median(|v_a - \hat{v_a}|)$, estimating standard deviation 
for a normal distribution.} 
In the checking phase, as long as the checker finds a value for a numerical 
attribute in the checked file outside of this interval, 
a warning would be printed to the user indicating the violating value, 
the attribute, and the upper or lower Hampel threshold. 

The intuition behind this is that if the user has input a value 
that falls outside of an interval containing values that are considered 
``normal'' over the entire sample dataset, 
that value will probably cause an error, in particular for performance. 
We cannot know for sure if this value will cause an issue. 
For instance, a user might have a machine with 
particularly high-end hardware, 
in which case a value beyond the upper Hampel threshold may be appropriate. 
