\section*{\app: Automated Verification of Configuration Files}
\paragraph{Overview:} 
Configuration errors (also known as misconfigurations) have become one of the major causes of system failures, resulting in security vulnerabilities, application outages, and incorrect program executions. 
Configuration errors were reported to be the largest fraction of failures in storage systems in a 2014 industry survey. 
Although many techniques have been proposed for configuration error detection, these approaches mainly can only be applied after an error has occurred.
We cannot simply apply techniques that exist for software verification: configuration files mainly consists of sequences of assigning some values to system variables.
Additionally, there is no formal specification describing properties of configuration files.

We propose a framework for automated verification of configuration files. We first automatically derive a specification, and the we check if a given configuration 
file adheres to that specification.
The main idea is to analyze a large number of configuration files from publicly available datasets and derive a corresponding specification. To do that we first develop a language model for configuration files. One question we 
need to address is what if there is an error in the given training set. Our proposed framework 
is general enough that it can still derive a correct specification, under 
the assumption that the same error does not appear in a significant amount of files.


\paragraph{Intellectual Merits:} This work's key intellectual contributions are:
\begin{itemize}
\item A design of a new framework that can learn a language model from an only partially correct set of configuration files
\item Investigating methods to handle the untyped and unstructured nature of configuration files
\item Finding techniques to optimize learning parameters to handle the probabilistic nature of the training set
\item Developing a modular tool for further work in configuration verification
\end{itemize}

\paragraph{Broader Impact:} 
Because configuration files are untyped, unstructured, and mostly undocumented, we are not directly leverage existing techniques for verification as used in programming languages.
Making progress towards verification in such environments, will open the door for work in other domain which have previously been to noisy to handle with existing techniques.
It would complement existing post-failure error diagnosis efforts, and could benefit other verification domains, such as network traffic verification. 
This could significantly reduce large scale system failures without adding any additional burden to users and system developers. 

