\section*{\app: Automated Verification of Configuration Files}
\paragraph{Overview:} 
Configuration errors (also known as misconfigurations) have become one of the major causes of system failures, resulting in security vulnerabilities, application outages, and incorrect program executions. 
Configuration errors were reported to be the largest fraction of failures in storage systems in a 2014 industry survey. 
Although many techniques have been proposed for a configuration error detection, these approaches mainly can be applied after an error has occurred.

We propose a framework for automated verification of configuration files. 
Verifying configuration files is different than verifying standard programs: configuration
files are missing a program structure and a specification. They are written in a
very low level programming paradigm: they mainly consists of sequences of assignments 
of some 
values to system variables. In such a framework, writing a specification is a complex task, 
since it is not clear what the specification should look like and which properties should 
it address. Our framework works as follows: in the pre-processing stage, 
we first automatically derive a specification. Once we have a specification, 
we check if a given configuration 
file adheres to that specification.
Deriving a specification is based on analyzing a large number of configuration files from publicly available 
datasets. We recognize patterns and keyword correlations and from those findings we derive 
a corresponding specification. To do that we first develop a language model for configuration files. One question we 
need to address is: what if there is an error in the given training set? Our proposed framework 
is general enough that it can still derive a correct specification, under 
the assumption that the same error does not appear in a significant amount of files.
We plan to evaluate our framework on real-world configuration files 
used in the TravisIC tool.

\paragraph{Intellectual Merits:} This work's key intellectual contributions are:
\begin{itemize}
\item A design of a new framework that can learn a language model from an only partially correct set of configuration files
\item Investigating methods to handle the untyped and unstructured nature of configuration files
\item Finding techniques to optimize learning parameters to handle the probabilistic nature of the training set
\item Developing a modular tool for further work in configuration verification
\end{itemize}

\paragraph{Broader Impact:} 
Our main goal is to help to reduce large scale system failures without adding any additional burden to users and system developers. 
Making progress towards verification of configuration files, will enable a static analysis of configuration files. This way potentially disastrous errors can be 
detected before they appear.
We envision \app to complement already existing post-failure error diagnosis efforts.
In addition, this work will also benefit other verification domains, such as network traffic verification. 

