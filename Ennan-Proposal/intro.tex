\section*{\app: Automated Verification of Configuration Files}
\paragraph{Overview:} 
Configuration errors (also known as misconfigurations) have become one of the major causes of system failures, resulting in security vulnerabilities, application outages, and incorrect program executions. 
Configuration errors were reported to be the largest fraction of failures in storage systems in a 2014 industry survey. 
Although many techniques have been proposed for configuration error detection, these approaches mainly can only be applied after an error has occurred.
Tools which detect misconfigurations at compile time often require the user to explicitly define constraints, a time consuming and error prone task.
We propose a framework to statically verify configuration files based on automatically generated constraints that define a correct file.
The core idea is to analyze many examples of configuration files from widely available datasets and derive a language model (a set of constraints) that can be used to verify new configuration files.

There are many datasets of configuration files online, but we only know that most are correct - some may be incorrect.
These files will be unlabelled so we do not know if a particular file is correct or incorrect.
Furthermore, both the syntax and semantics of configuration files are notoriously underspecified
Since we are building a general tool to work with any configuration language, \app must be able to handle probabilistic learning in many dimensions.

\paragraph{Intellectual Merits:} This work's key intellectual contributions are:
\begin{itemize}
\item A design of a new framework that can learn a language model from an only partially correct set of configuration files
\item Investigating methods to handle the untyped and unstructured nature of configuration files
\item Finding techniques to optimize learning parameters to handle the probabilistic nature of the training set
\item Developing a modular tool for further work in configuration verification
\end{itemize}

\paragraph{Broader Impact:} 
Because configuration files are untyped, unstructured, and mostly undocumented, we are not directly leverage existing techniques for verification as used in programming languages.
Making progress towards verification in such environments, will open the door for work in other domain which have previously been to noisy to handle with existing techniques.
It would complement existing post-failure error diagnosis efforts, and could benefit other verification domains, such as network traffic verification. 
This could significantly reduce large scale system failures without adding any additional burden to users and system developers. 

