
\section{Broader Impact}

Since her beginning of teaching at Yale, the PI is actively including topics from her 
research into her courses (such as ``Software Analysis and Verification''). This way
she motivates undergraduate students to get involved in the research. Until now 
she supervised more than 20 undergraduate student projects. Two students (Reinking, Cai) 
who did an undergraduate research projects with her, are now PhD students at 
UC Berkeley, and two more students are applying (Anklessaria, La) to graduate schools 
this year. Additionally, there are two student startups, which are now supported
by YCombinator, that started under her supervision. Those startup are PatientBank
and Py, an app for learning Python.

The PI will continue to involve the undergraduate student into 
this project as well, spreading this way the project ideas to a broader community.
In addition, we plan to disseminate the results via publications and giving talks 
to the communities of PL/Verification researchers and System researchers.

We believe that \app could have the following broader impacts:

\begin{itemize}

\item {\bf Enabling language-based configuration verification.}
  Today's misconfiguration checking approaches are still using {\em ad hoc}
  algorithms to detect potential misconfiguration root causes. \app 
  should enable more rigorous verification of configuration files. By verification we
  mean checking that the given configuration file adheres to a
  specification.
%\item {\bf Preventing configuration errors without adding any additional
%  burden to users.} Because \app employs automatic machine learning 
%  algorithms to extract rules, users or system administrators do not need
%  to manually write their specifications, which significantly reduce the
%  burden of users and system administrators. Such a way not only
%  makes the configuration verification process become much easier and more 
%  usable, but also decreases the potential issues of writing 
%  specifications.

\item {\bf Complementing post-failure error diagnosis efforts.}
  There have been many diagnosis and troubleshooting efforts to localize
  the root causes of configuration errors after system failures occur.
  We call these approaches as post-failure error diagnosis efforts.
  \app will offer a pre-failure error checking
  which could be looked as a preliminary checking before installing
  systems. Because \app is able to find out many sophisticated 
  configuration errors, it removes many labors of 
  the post-failure error diagnosis tools. In addition, the results
  output by \app could be used as a guidance for all the post-failure
  error diagnosis approaches, making these approaches more efficient.

\item {\bf Benefiting other communities, \eg, network misconfiguration.} 
  Besides software misconfiguration, network community has the same 
  misconfiguration problems. In particular, many recent service outage 
  reports exposed network-level configuration error will lead to
  very critical service outages. \app could be extended to reason about
  network-level configuration errors.

\end{itemize}

