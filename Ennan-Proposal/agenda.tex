
\section{Research Agenda}

The primary work involved in this project
falls into three broad categories:
(i) defining a more comprehensive language model,
(ii) developing new learning algorithms, and
(iii) developing and evaluating \app
  in real-world configuration files.

\subsection{Defining A More Comprehensive Language Model}

While we feel our proposed language model to be relatively comprehensive
to capture many tricky misconfiguration cases,
we will need to define a better language model to represent more types of
configuration files. The configuration files of some of software systems, 
\eg, squid, are not limited to the key-value parameters; thus,
we need to extend the current language model in order to represent
the configuration files of these systems.
In addition, some other types of latent configuration errors 
have been exposed by recent work~\cite{xu16early}.
For example, the path parameter in a MySQL configuration file
has an unknown user account. Such an error cannot be represented in
our proposed language model. Thus, we need to improve it to capture
more error types.

\subsection{Proposing New Learning Algorithms}

We are planning to develop new learning algorithm capable of extracting
rules more accurately and efficiently. We plan to first investigate 
some existing learning algorithms for learning program rules, \eg,
Raychev {\em et al.}~\cite{raychev15predicting, raychev16learning},
and then adapt these algorithms to our scenario, \ie, learning 
the configuration file rules.

\subsection{Deploying and Evaluating \app In Real-World Services}

The project's third main task is to prototype \app and make \app
practical to real-world misconfiguration verification,
thus making \app become a useful tool in practice.
We plan to evaluate the \app prototype's efficiency, usability, and
accuracy using a variety of micro- and macrobenchmarks.
For example, we can evaluate the efficiency and accuracy of 
each module of \app, thus understanding \app's practicality.
More importantly, we will use \app to detect real-world configuration
files~\cite{configdataset}, and then manually check whether \app is able
to successfully detect potential configuration files.
