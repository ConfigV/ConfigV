
\section{Extending \app Via Version Space Learning}
\label{sec:travis}

The current design of \app does not have any assumptions on whether the
configuration files in the training sets are correct or 
incorrect, thus making \app more general and practical.
Nevertheless, some cases in practice 
may allow us to assume some of configuration 
files in the training sets are guaranteed to be correct.
For example, TravisCI is a famous system connected 
to Github, and it allows programmers to automatically run 
their test suites on every code commit. 
Thus, a TravisCI user needs to add a configuration file 
to the repository that 
enables TravisCI and specifies build conditions, 
such as which dependencies are required, 
and a set of benchmarks to test. This ensures 
the tool to be automatically built correctly on a fresh machine.
Such a configuration file management scheme, in fact, guarantees 
the configuration file versions in the commit history to be 
correct -- otherwise, TravisCI should not work correctly in the past.
\ennan{Need one or two sentences to describe TravisCI misconfiguration
problem is important.}

Driven by the above motivating example, 
we propose an extension to \app that
learns a given training set with pre-knowledge that 
this training set contains some configuration files 
guaranteed to be correct.
Such an extension employs an existsing technique from the machine learning community, named version space learning,
to significantly decrease the false positive rate while maintaining 
the detection capability of \app.

\subsection{Version Space For Verification}

Version space learning builds a logical constraint model for binary
classification, which we use to test a configuration file for membership
in the set of all correct files~\cite{mitchell82}.
Traditional version space learning builds a model that tests membership
using a series of disjunctions from a set of predefined hypotheses.
In the extended \app, rather than describing a correct file by allowable 
traits, we describe the required traits. 
This way, our system can not only flag misconfigurations, 
but also give the points of failure for non-membership.
We now show in detail how this process works.

Our definition of a configuration file, $C$, is a file that can be
transformed to an ordered list of $(keyword, value)$ pairs, called
$Line$s. \app builds a model for a single file, $M(C)$, 
that is the set of all possible relations between the lines in the file.
A relation is described by user provided templates.
A template is a function taking some number of lines and determining 
a relation on the keywords of 
the lines ($Line^{n} \rightarrow Rel_{k1,..,kn}$).
For example, we may derive the relation \texttt{extension mysql.so} comes before \texttt{extension recode.so} by using the ordering template.
ConfigC features the following default templates.

\begin{center}
\begin{tabular} {|c|c|c|}
 \hline
 Template & Input  & Relation \\
\hline
\hline
 ordering & (Line,Line) & Before $\vert$ After $\vert$ None \\
 \hline
 integer relation & (Line,Line)  & $< | == | >$  \\
\hline
 type & (Line,Line)  & String $\vert$ Int $\vert$ Filepath $\vert$ IP  \\
\hline
 missing entry & (Line,Line) & Required $\vert$ Not \\
\hline
\end{tabular}
\end{center}

It is assumed if a file is correct, all relations in that file are in the set of necessary relations for any other file to be correct ($Correct(C) \implies \forall r \in M(C), r \in Nec$).
In this way, the initial model is built by creating the strongest conditions for a correct file, called the \textit{specific boundary} in version space learning.
This model is then iteratively relaxed as more examples are seen, a process called \textit{candidate elimination}.
Since we maintain the strongest condition for correctness, this approach will identify many correct files as incorrect, which we call a high false positive rate.
This is a problem when a user is then asked to manually review many files for errors, when the files are in fact correct.

\subsection{Learning From Temporal Properties} 

%ConfigC can provide justifications, but has a high false positive rate.
%However, ConfigC only analyzes correct configuration files, and does not consider any ordering between these correct files.
We now present a new algorithm to decrease the false positive rate 
by learning on both correct and incorrect examples, 
as well as temporal structure of these examples.
This extension will build a logical formula representing the entire 
history of examples, and use an SMT solver to find a classification model.

Since a configuration file (\eg, TravisCI configuration file) is dependent
on the code it is trying to build, 
we must consider a more general sense of configuration file.
We will call this a program summary $P_t$, which is a representation 
of the repository which contains the information 
relevant to the learning process.
The subscript on $P_t$ is a timestamp tag based on 
the ordered git commit history.
In the case of TravisCI, this include the \verb|.travis.yml| file, 
as well as code features that may effect build status, 
such as programming language and a list of imported libraries.
The summary must be \textit{sufficiently detailed}, that is it must contain every piece of information that might lead to a build error.

%IMPORTANT, but no space
%However, a git history is not a limited to a single linear timeline.
%Git features the ability to \textit{branch}, which allows to simultaneous commit chains.
%To handle the start of a branch, add a superscript to indicate the branch, and restart the counter on a branch.
%To handle the merge of two branches $P_{t}^{x}$ and $P_{t'}^{y}$, step to $P_{t+1}^{x}$, where $x$ is the mainline branch.
%We then say that $P_{t'}^{y}$ has no successor commit $P_{t'+1}^{y}$.

From this summary we can then build a model $M(P_t)$, as in \app, 
which is the full set of possible relations derivable 
from the program summary.
Again, the user must provide templates for the learning process.

We will denote the build status returned from TravisCI when run on $P_t$ with $S(P_t)$.
Any files with a $Pass$ build status are correct, and any files with an $Err$ build status are incorrect.
%In this application, we consider all non-erroring build status to be passing, denoted $Pass$ and otherwise $Err$.
For brevity, we denote sequences of build statuses with the following notation:
\begin{align*}
  S(P_t)=Pass \land S(P_{t+1})=Err \implies S(P_{t,t+1}) = PE
\end{align*}

We now consider both incorrect and correct 
and so must introduce the \textit{general boundary}.
The general boundary is the dual of the specific boundary, and is the most relaxed requirement for a positive classification.
We denoted specific boundary as the set of necessary relations $Nec$, and now denote the general boundary as the set of breaking relations $Br$.
With this notation, we can formally express the requirement that the program summary is sufficiently detailed.
\begin{align}
  \forall S(P_t)=Err, \exists r \in M(P_t), r \in Br \label{eq:E1}
\end{align}

From the above we know that if a build is erroring, then there must exist at least one error.
%By pushing the negation into the formula,
We can also know that if a build is passing, then there must not exist any errors.
That is, the model of a passing commit must not contain any rules which are breaking.
Note we are not, however, guaranteed that any rules from a passing commit are necessary.
\begin{align}
  S(P_t) = Err \implies \exists r \in  M (P_t), r \in Br \label{eq:E}\\
  S(P_t) = Pass \implies \forall r \in  M (P_t), r \notin Br \label{eq:P}
\end{align}

%While Eq. \ref{eq:E} and \ref{eq:P} might build a basic model, they will do not capture all of the available knowledge.
%The key insight is that 
Additionally, when we commit a break ($PE$), we can localize the error to one of the relations that changed.
Either we removed something that was necessary, or added something that was breaking.
Note that this is an inclusive disjunction, since a erroring commit can break multiple things at once.
Expressed formally, where $\setminus$ is the set difference, that is:
\begin{align}
  S(P_{t,t+1}) &= PE \implies \nonumber \\
  \exists& r \in (M(P_{t})\ \setminus M(P_{t+1})), r \in Nec\ \nonumber \\
  \lor \ \exists& r \in (M(P_{t+1}) \setminus M(P_{t})), r \in Br \label{eq:PE}
\end{align}

We then can combine Eq \ref{eq:E}, \ref{eq:P}, and \ref{eq:PE} with
conjunctions and ask a SMT solver for a model satisfying the formula.
The resulting model will be the sets $Nec$, and $Br$, which can be used to check new configuration files.
Since we used a similar model to \app, we will still be able to provide justifications for the classification results.

