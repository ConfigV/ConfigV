\section{Our Previous Work: ConfigC}
\label{sec:prelim}

We have proposed and developed a preliminary system, named
ConfigC~\cite{santolucitoCAV}, which is (to the best of
our knowledge) the first systematic effort capable of automatically
verifying configuration files before the configuration files
are deployed and installed.
In particular, ConfigC is a framework that first automatically
analyzes datasets of correct configuration files, 
and then derives rules for building a language model 
from the give datasets.
Finally, the resulting language model could be used to verify 
new configuration files (called target configuration files) and 
detect errors in these target configuration files.
The configuration errors ConfigC can deal with cover:
entry missing errors, ordering errors, type errors and
value correlation.
To our knowledge, there is no existing effort that is able to detect
the above errors, because they are very tricky to identify in
practice.

ConfigC can detect the above errors by overcoming 
main technical challenges against automatic configuration verification.
First, because writing specifications for configuration verification
is difficult -- especially for specifying the above
tricky configuration errors, \eg, missing entry and ordering errors,
ConfigC employs a set of machine learning algorithms to
automate the specification writing. Machine learning based approach
employed by ConfigC enables the process of specification writing to become
automatic. As long as a set of configuration files are provided,
ConfigC can automatically generate a set of rules that could be used
as specifications.
Second, because existing configuration files do not have any language
structures and grammar, it is difficult to verify configuration files.
ConfigC proposes a new language model and transforms target configuration
files into the proposed language model. In other words, ConfigC uses
the language model as a uniform representation and parses the 
configuration files to verify into such a representation.
With the uniform representation and specifications in hand,
ConfigC is able to verify whether the given configuration files
meet the specification.

\paragraph{Evaluations on ConfigC.}
We implemented a prototype system by following 
our ConfigC design~\cite{santolucitoCAV}.
In order to demonstrate the capability of ConfigC,
we evaluated the ConfigC prototype based on real-world
MySQL configuration files~\cite{configdataset} 
containing tricky misconfiguration errors,
such as entry missing errors, ordering errors, and
value correlation errors.
We extracted 20 MySQL configuration files to detect, 
and grouped them into four categories according to their
error types: missing entry, type error, ordering error
and value correlation.
Table~\ref{table:res} shows the evaluation results we use
ConfigC to detect configuration errors in these 20 real-world
configuration files.
As shown in Table~\ref{table:res},
ConfigC is able to successfully report the misconfiguration problems.

\begin{table}[h]
\centering
\caption{Evaluation results on misconfiguration detection of ConfigC}
\label{table:res}
\begin{tabular}{|l|l|l|}
\hline
Error Type       & Passing Tests & False Positives  \\ 
\hline
\hline
Missing Entry      & 5/5           & 1, 0, 0, 0, 4        \\ \hline
Type Error         & 5/5           & 0, 0, 0, 0, 0          \\ \hline
Ordering Error     & 5/5           & 0, 2, 1, 0, 6       \\ \hline
Value Correlation  & 4/5           & 0, 0, 0, 1, 0        \\ 
\hline
\end{tabular}
\end{table}

\paragraph{Limitations.}
ConfigC still has the following limitations.

\begin{enumerate}

\item ConfigC requires the datasets of configuration files for training 
  have to be correct, which is hard to achieve in practice, since 
  it is difficult for administrators or users to offer a
  set of 100\% correct configuration files for training.

\item The language model of ConfigC is still a starting point. It
  is hard to formulate a uniform representation for diverse
  configuration files.\ennan{we still need to put one more sentence
  here to distinguish why our current language model is better
  than ConfigC's language model}

\item ConfigC can only offer coarse-grained value correlation errors,
  \eg, $\texttt{max\_connections} > \texttt{mysql.max\_persistent}$.
  In practice, many software outages were caused by more tricky, called
  fine-grained value correlation~\cite{correlation}. For example, 
  in a MySQL configuration file, $\texttt{key\_buffer\_size}$ should be 
  higher than $\texttt{max\_connections} \times 
  \texttt{sort\_buffer\_size}$~\cite{correlation};
  otherwise, the MySQL load will be very high.

\item ConfigC cannot check singular value anomalies. For example,
  a parameter relating to memory in MySQL configuration files is
  setted too large, exhausting the RAM and causing extreme slowness
  or even a crash. Such an anomalous value does not violate any
  constraints (\eg, type error and value ordering error), but 
  it makes system behave wrongly.

\end{enumerate}

This proposal aims to address the above limitations. 
